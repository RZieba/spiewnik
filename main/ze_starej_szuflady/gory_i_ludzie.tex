\section{Góry i ludzie}{sł. J. Harasymowicz, muz. tradycyjna}
\begin{text}
Góry i ludzie z nieba schodzą,
Trochanowscy prowadzą basy.
Przez wieś wiodą jak niedźwiedzia,
Jak jastrząb skrzypeczki płaczą.

To Sławko gra skrzypce trzyma,
smykiem zahacza o szczyty dalsze.
A bas Piotra jak wicher wydyma
banie na cerkwi na cerkwi w Bielance.

Góry i ludzie z nieba schodzą,
Na drodze życzliwa życzliwa ciemność.
Wreszcie po latach tych przy stole
zsiadło się ze mną morze Łemków.

Jedna nad nami Łemkowyna
i jeden Święty Jerzy czuwa.
Jezus na tronie wypoczywa,
czesany w koki jak Samuraj.
\end{text}
\begin{chord}
E F#m
A E
E F#m
A E

E F#m
A E
E F#m
A E

E F#m
A E
E F#m
A E

E F#m
A E
E F#m
A E
\end{chord}