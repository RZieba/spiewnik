%%
%% Author: bartek.rydz
%% 28.05.2018
%%
% Preamble
\tytul{Postój zimowy}{}{Robert Kasprzycki}
\begin{text}
    A kiedy ręka uchyla zasłony,\\
    I świat po nocy jawi się widzialny,\\
    Na śnieżnych wieżach kołyszą się dzwony\\
    I stoją w oknach szronu białe palmy,\\
    I puch opada pazurkiem strącony.

    Łaskawa zima codziennie nam sprzyja\\
    Pośrodku ogni niepewnego wieku,\\
    Zwierzęta dziwne mróz w niebie rozwija,\\
    Gwiazda osiada na biednym człowieku\\
    I tając grzeje, już blisko Wigilia.

    Idziemy w nasze wysokie pokoje,\\
    Dzwonki czy skrzydła za oknami brzęczą\\
    I słońce pada, siostrzyczko, na twoje\\
    Włosy, i zjeżdża jak chłopak poręczą,\\
    Ale ty inną jasność masz na czole.

    Przyjmij ode mnie tę gałązkę małą\\
    Mimozy, w nasze przywiezionej śniegi.\\
    Bo gdyby ciebie wszystko nie kochało,\\
    Tobym nie myślał, zamknąwszy powieki,\\
    Że piękne jest to, o czym się milczało.

    O Boże, jak niepewne nasze losy,\\
    Jaka potężna siła nami toczy,\\
    Ile złych godzin serca nam spustoszy,\\
    Zanim ty, wierna, zamkniesz moje oczy,\\
    Ty, przychodząca do mnie szarym rankiem,\\
    Jak matka senna z kopcącym kagankiem.

    Ogromne wody, zamglone stolice,\\
    Mroźny znak wojen, co w niebie się pali -\\
    Ale ja niczym siebie nie nasycę\\
    I tak oboje będziemy czekali\\
    Na promień ostry, który nas otwiera,\\
    Nie wiem, czy gdy się żyje, czy umiera.

    Zimo dobra, bielą otul nas,\\
    Bo każda nasza chwila przebudzenia czeka,\\
    Z dawnych smutków oczyść naszą twarz,\\
    Bo mamy jechać razem, a droga daleka.\\
    I niech się spełni złotej łaski czas.
\end{text}
\begin{chord}

\end{chord}