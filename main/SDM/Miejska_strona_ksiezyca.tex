%%
%% Author: bartek.rydz
%% 17.02.2019
%%
% Preamble
\tytul{Miejska strona księżyca}{sł. A. Ziemianin, muz. K. Myszkowski}{Stare Dobre Małżeństwo}
\begin{text}
    Moje miasto zimą strasznie kaszle\\
    W piwnicach ma węgiel wodę i szczury\\
    Wciąż więcej tu powodów do strachu\\
    A coraz mniej do dumy

    Dziewczyna na Plantach dziwna jakaś\\
    Z domu uciekła pierwszym lepszym pociągiem\\
    Złapać próbuje atmosferę miasta\\
    Do życia ją ciągnie bokiem

    Moje miasto wiosną też jest straszne\\
    Choć można żyć już chyba bardziej\\
    Tu wszystko zawsze ma swoją cenę\\
    Słońce na rynku monety kładzie

    Dziewczyna na Plantach skubie ptaka\\
    Uciekła z miasteczka bo ją znudziło\\
    Gołąb nie gołąb może być kawka\\
    Za chwilę już będzie miło

    Moje miasto latem ma sucho w gardle\\
    Ale przecież żyć jakoś trzeba\\
    Księżyc twarz ukazuje miejską\\
    Innego wyjścia już nie ma

    Dziewczyna na Plantach w siódmym niebie\\
    Miasteczko śni się jeszcze czasem\\
    Makijaż twarzy mocno się trzyma\\
    Noce są takie jasne

    Moje miasto jesienią już lekko gaśnie\\
    I wszystko jakby powoli się kurczyło\\
    Już wcześniej nawet płoną latarnie\\
    W knajpie kończy się piwo

    Dziewczyna na Plantach trochę osowiała\\
    Z miasteczka już nic nie pamięta\\
    Miejski księżyc zagląda jej w oczy\\
    Za chwilę będą chyba święta
\end{text}
\begin{chord}
    Fis_{7}^{9}\\
    Fis_{7}^{9}\\
    H^{7}\\
    Fis_{7}^{9}

    Cis^{7} H^{7}\\
    Cis^{7} H^{7}\\
    Cis^{7} H^{7}\\
    Fis_{7}^{9} Cis^{7}
\end{chord}