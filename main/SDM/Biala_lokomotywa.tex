%%
%% Author: bartek.rydz
%% 16.02.2019
%%
% Preamble
\tytul{Biała lokomotywa}{}{Stare Dobre Małżeństwo}
\begin{text}
    Sunęła poprzez czarne łąki\\
    Sunęła przez spalony las\\
    Mijała bram zwęglone szczątki\\
    Płynęła przez wspomnienia miast\\
    Biała lokomotywa\\
    Biała lokomotywa

    Skąd wzięła się w krainie śmierci\\
    Ta żywa zjawa istny cud\\
    Tu pośród pustych marnych wierszy\\
    Tu gdzie już tylko czarny kurz\\
    Biała lokomotywa\\
    Biała lokomotywa

    Ach czyj ach czyj to jest\\
    Ten piękny hojny gest\\
    Kto mi tu przysłał ją\\
    Bym się wydostał stąd\\
    Białą lokomotywę\\
    Białą lokomotywę

    Ach któż, ach któż to może być\\
    Beze mnie kto nie może żyć\\
    I bym zmartwychwstał błaga mnie\\
    By mnie obudził jasny zew\\
    Białej lokomotywy\\
    Białej lokomotywy

    Suniemy poprzez czarne łąki\\
    Suniemy przez spalony las\\
    Mijamy bram zwęglone szczątki\\
    Płyniemy przez wspomnienia miast\\
    Z Białą lokomotywą\\
    Z Białą lokomotywą

    Gdzie brzęczą pszczoły pluszcze rzeka\\
    Gdzie słońca blask i cienie drzew\\
    Do tej co na mnie w życiu czeka\\
    Do życia znowu nieś mnie nieś\\
    Biała lokomotywo\\
    Biała lokomotywo
\end{text}
\begin{chord}
    G C\\
    D C G\\
    G C\\
    D C G D D^{7}\\
    C D G\\
    C D G

    G C\\
    D C G\\
    G C\\
    D C G D D^{7}\\
    C D G\\
    C D G

    G\\
    G\\
    G\\
    G^{7}\\
    C D G\\
    C D G
\end{chord}