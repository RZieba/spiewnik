%%
%% Author: bartek.rydz
%% 19.02.2019
%%
% Preamble
\tytul{Mówi mądrość}{sł. Jan Rybowicz, muz. Krzysztof Myszkowski}{Stare Dobre Małżeństwo}
\begin{text}
    Nie zmienisz ani jednej joty w zakonie świata,\\
    Nie wymażesz ani jednej kropki nad 'i'\\
    Stare grzechy rzucają długie cienie, które cię obraczają\\
    W twojej drodze ku światłości.\\
    Stare kłamstwa powracają wciąż,\\
    Nie miałeś racji, przyznaj się - szydzą

    \vin I tylko ty sam, i tylko ty sam\\
    \vin Zmieniłeś się znów, po raz drugi\\
    \vin I tylko ty sam czujesz się wciąż\\
    \vin niepotrzebny nikomu, tylko Ty sam...

    Skrzywdzeni przez Ciebie ludzie domagają się wciąż\\
    Rehabilitacji\\
    Już nawet nie ludzie, tylko ich cienie,\\
    To była wtedy moja wina - mówisz w pustkę\\
    I masz nadzieje, że ktoś tam to jednak usłyszy,\\
    Rozgrzeszy Cię, uwolni od ciężaru niesławnej pamięci.

    Kanon, który chciałeś naruszyć, pozostał nienaruszony\\
    I nawet już nie drwi z Twej donkiszoterii\\
    I wiesz po otwarciu wszystkich drzwi, że są zawsze jeszcze jedne\\
    Drzwi nie do odemknięcia.\\
    Obraz jest piękny, z określonej odległości,\\
    Z innych staje się tylko nieczytelną plamą

    Nie zmienisz ani jednej joty w zakonie świata,\\
    Nie wymażesz ani jednej kropki nad 'i'\\
    Wróciłeś do źródła bogatszy o cierpienie\\
    I uspokojony cierpieniem\\
    Mówisz: nie ma rzeczy niepotrzebnych,\\
    Wszystko ma swój czas, miejsce i rolę do spełnienia.
\end{text}
\begin{chord}
    a F a\\
    F e a\\
    a F a\\
    F e a\\
    a F a\\
    F e a

    C F C\\
    G a F G\\
    C F C\\
    G a F G C
\end{chord}