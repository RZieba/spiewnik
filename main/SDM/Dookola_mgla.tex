%%
%% Author: bartek.rydz
%% 19.02.2019
%%
% Preamble
\tytul{Dookoła mgła}{sł. E. Stachura, muz. K. Myszkowski}{Stare Dobre Małżeństwo}
\begin{text}
    Jak długo pisana mi jeszcze włóczęga?\\
    Ech gwiazdo - ogniku ty błędny mych dni.\\
    Spraw, by skończyła się wreszcie ta męka.\\
    I zapędź, do czułych zakulaj mnie drzwi!

    Lecz gdzie jest ten dom, jak tam idzie się doń?\\
    Gdzie jest ta stanica, gdzie progi te są?\\
    Tam most jest na rzece, za rzeką jest sad;\\
    Tam próżnia się kończy, zaczyna się świat!

    Lecz gdzie rzeka ta, gdzie rzucony jest most?\\
    Gdzie sad ten jest biały, jabłonki gdzie są?\\
    Na drzewach owoce i strąca je wiatr,\\
    Do kosza je zbiera ta ręka jak kwiat.

    Te strony gdzieś są, gdzieś daleko za mgłą,\\
    Więc idę i dalej przedzieram się wciąż.\\
    Zbierają się ptaki, ruszają na szlak,\\
    Już lecą, wprost lecą, nie błądzą jak ja.

    Jak długo pisana mi jeszcze włóczęga?\\
    Ech gwiazdo - ogniku ty błędny mych dni.\\
    Spraw, by skończyła się wreszcie ta męka.\\
    I zapędź, do czułych zakulaj mnie drzwi!
\end{text}
\begin{chord}
    e D C^{7} e\\
    C D G D\\
    C G a h^{7}\\
    C a h^{7} e
\end{chord}