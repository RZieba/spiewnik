%%
%% Author: bartek.rydz
%% 19.02.2019
%%
% Preamble
\tytul{Na stacji Jerzego z Podebrad}{sł. muz. Jaromir Nohavica}{Zbigniew Zamachowski}
\begin{text}
    Widzimy się co dzień, na schodach w metrze,\\
    gdy Ona jedzie na dół – a ja na powierzchnię...\\
    Ja wracam z nocnej zmiany,\\
    Ty pracujesz rano;\\
    Ja jestem niewyspany,\\
    Ty z twarzą zatroskaną...

    A schody jadą, choć mogłyby stać,\\
    na stacji Jerzego z Podiebrad...

    Praga, o szóstej, sennie jeszcze ziewa\\
    i tylko my naiwni, robimy, co trzeba...\\
    Ja spieszę się z kliniki,\\
    gna do kiosku Ona;\\
    Zmęczone dwa trybiki,\\
    dwie wyspy wśród miliona...

    A schody jadą, choć mogłyby stać,\\
    na stacji Jerzego z Podiebrad...

    Choć o tej samej porze - randki są ruchome,\\
    bo w tym tandemie każdy jedzie w swoją stronę\\
    Ja w lewo, ona w prawo\\
    nie ma odwrotu\\
    Ją czeka 'Rude Pravo',\\
    a na mnie pusty pokój.

    A schody jadą, choć mogłyby stać,\\
    na stacji Jerzego z Podiebrad...

    Na czarodziejskich schodach, czuję w sercu drżenie,\\
    gdy kioskareczka Ewa śle mi swe spojrzenie.\\
    W pośpiechu ledwie zdążę\\
    szepnąć: „Witam, z rana”,\\
    bo całowania, w biegu,\\
    surowo się zabrania!

    A schody jadą, choć mogłyby stać,\\
    na stacji Jerzego z Podiebrad...

    A Praga drzemie i nic jeszcze nie wie\\
    o dwojgu zakochanych, zapatrzonych w siebie...\\
    Już tęsknią nasze włosy,\\
    w pędzie poplątane,\\
    Do tego, co nas czeka...\\
    Do tego, co nieznane...

    A schody jadą, choć mogłyby stać,\\
    na stacji Jerzego z Podiebrad...
\end{text}
\begin{chord}

\end{chord}