%%
%% Author: bartek.rydz
%% 12.02.2019
%%
% Preamble
\tytul{Życie to nie teatr}{sł. E. Stachura, muz. J. Różański}{Jacek Różański}
\begin{text}
    Życie to jest teatr, mówisz ciągle, opowiadasz;\\
    Maski coraz inne, coraz mylne się zakłada;\\
    Wszystko to zabawa, wszystko to jest jedna gra\\
    Przy otwartych i zamkniętych drzwiach to jest gra!

    Życie to nie teatr, ja ci na to odpowiadam;\\
    Życie to nie tylko kolorowa maskarada;\\
    Życie jest straszniejsze i piękniejsze jeszcze jest;\\
    Wszystko przy nim blednie,\\
    blednie nawet sama śmierć!

    Ty i ja - teatry to są dwa, Ty i ja!\\
    Ty - ty prawdziwej nie uronisz łzy.\\
    Ty najwyżej w górę wznosisz brwi.\\
    Nawet kiedy źle ci jest, to nie jest źle, bo ty grasz!\\
    Ja - duszę na ramieniu wiecznie mam.\\
    Cały jestem zbudowany z ran.\\
    Lecz kaleką nie ja jestem, tylko ty!

    Dzisiaj bankiet u artystów, ty się tam wybierasz;\\
    Gości będzie dużo, nieodstępna tyraliera;\\
    Flirt i alkohole, może tańce będą też,\\
    Drzwi otwarte zamkną potem się, no i cześć!

    Wpadnę tam na chwilę, zanim spuchnie atmosfera;\\
    Wódki dwie wypiję, potem cicho się pozbieram;\\
    Wyjdę na ulicę, przy fontannie zmoczę łeb;\\
    Wyjdę na przestworza, przecudowny stworzę wiersz.

    Ty i ja - teatry to są dwa, Ty i ja!\\
    Ty - ty prawdziwej nie uronisz łzy.\\
    Ty najwyżej w górę wznosisz brwi.\\
    I niezaraźliwy wcale jest twój śmiech, bo ty grasz!\\
    Ja - duszę na ramieniu wiecznie mam.\\
    Cały jestem zbudowany z ran.\\
    Lecz gdy śmieje się, to wkrąg się śmieje świat!
\end{text}
\begin{chord}
    a E\\
    E^{7} a\\
    F C\\
    G C G

    a E\\
    E^{7} a\\
    F C\\
    E^{7}\\
    a C^{7}

    F G C G\\
    C E^{7} a\\
    C^{7} F\\
    G C G\\
    C E^{7} a\\
    C^{7} F\\
    G C G

    a E\\
    E^{7} a\\
    F C\\
    G C G

    a E\\
    E^{7} a\\
    F C\\
    E^{7} a C^{7}

    F G C G\\
    C E^{7} a\\
    C^{7} F\\
    G C G\\
    C E^{7} a\\
    C^{7} F\\
    G C G
\end{chord}