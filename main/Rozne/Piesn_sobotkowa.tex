%%
%% Author: bartek.rydz
%% 10.02.2019
%%
% Preamble
\tytul{Pieśń sobótkowa}{}{Orkiestra Świętego Mikołaja}
\begin{text}
    Koło Jana, koło Jana -\\
    Koło Jana, koło Jana -

    Koło Jana, koło Jana\\
    - Tam dziewczęta się schodziły\\
    Sobie ogień nałożyły,\\
    Tam ich północ ciemna naszła.

    Nocel mała, kopiel moja\\
    - Tam dziewczęta się schodziły\\
    Sobie ogień nałożyły,\\
    Tam ich północ ciemna naszła...

    Koło Jana, koło Jana,\\
    Tam dziewczęta się schodziły,\\
    Sobie ogień nałożyły,\\
    Tam ich północ ciemna naszła.

    Tam na górze ogień gore,\\
    Na tej górze dwoje drzewa.\\
    Jedno drzewo, boże drzewo,\\
    Na tym drzewie kolebeczka.

    W tej kolebce panna Hanna,\\
    Kołychali, dwa braciszki.\\
    Kołychali, rozbombali,\\
    Rozbombali - wyrzucili.

    Padła Hanna licem w ziemię,\\
    Licem w ziemię, sercem w krzemię.\\
    I zapyta się braciszek,\\
    Czy nie zdrowiej, czy nie lepiej.

    Ona mówi: Mój braciszku,\\
    Główka boli, serce mdleje,\\
    A jak umrę, moi bracia,\\
    Pochowajcie w ogródeczku

    I nasiejcie trzy ziółeczek,\\
    A jak się zjadą panowie\\
    Będą tę ziemię łamali,\\
    Będą też mnie wspominali:

    Tutaj leży panna Hanna,\\
    Panna Hanna, siostra nasza -\\
    Nocel mała, kopiel moja\\
    - Nocel mała, kopiel moja

    Nocel mała, kopiel moja\\
    - Nocel mała, kopiel moja\\
    Nocel mała, kopiel moja\\
    - Nocel mała, kopiel moja...
\end{text}
\begin{chord}
    d C\\
    g a d

    d\\
    C\\
    g\\
    a d
\end{chord}