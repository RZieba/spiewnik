%%
%% Author: bartek.rydz
%% 10.02.2019
%%
% Preamble
\tytul{Burza}{sł. muz. Maciej 'Krążek' Służała}{Krążek}
\begin{text}
    Znów chmury zmęczone przysiadły na grani,\\
    Ciężkimi brzuchami pieszczą górskie grzbiety.\\
    Już drzewa jęknęły, smagnięte wichrami.\\
    I trawy przerażone knują głośne szepty.\\
    Umilkły już ptaki w gniazdach cicho siedzą.\\
    Świat czeka w napięciu z licem pociemniałym.\\
    Zbudziły się biesy z driadami tańczą.\\
    Już błysk srebrnej nici zalśnił oniemiały.\\
    Pierwsze ciężkie krople dopadły już ziemi.\\
    Jak ślepi młocarze chcą powalić wszystko.\\
    Już strumienie wody leją się za nimi,\\
    Szorując noc trokami spływają z gór szybko.\\
    Nowa srebrna nić przecięła czerń nieba.\\
    Echo przerażone ryczy dzikim grzmotem.\\
    Drzewa w wiatru graniu szamoczą się z deszczem,\\
    Jak rycerze antyczni, zmagając się z sobą.\\
    Jak turkot karety znów grzmot się przewalił.\\
    I znów atakują kropli wściekłych fale.\\
    A te co dopadły straciwszy swą siłę,\\
    Strumieniem wezbranym z gór odpływają.\\
    I fala za falą i tną jak lawina.\\
    Lecz sił już nie starcza kroplom i wiatrom.\\
    I nagle blask słońca przez chmury przenika.\\
    I tęcza w deszczowych drobinkach jak flaga.\\
    I cisza i spokój i znów świergot ptaka\\
    I słońce szaleje wśród zroszonej trawy\\
    I drzewa co z włosów wytrzepują krople\\
    I wszystko się cieszy po deszczowej kurzawie.\\
    I cisza
\end{text}
\begin{chord}
    a\\
    G\\
    F\\
    E
\end{chord}