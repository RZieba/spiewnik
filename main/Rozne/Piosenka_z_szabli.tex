%%
%% Author: bartek.rydz
%% 10.02.2019
%%
% Preamble
\tytul{Piosenka z szabli}{muz. Paweł Orkisz}{Paweł Orkisz}
\begin{text}
    Niech w księgach wiedzy szpera rabin\\
    nauka to jest wymysł diabli\\
    mądrością moją jest karabin\\
    i klinga ukochanej szabli

    Nie dbam o szarże, ni o gwiazdkę,\\
    co kiedyś mi przystroją kołnierz\\
    wy piszcie klechdy i powiastki\\
    ja biję się jak musi żołnierz

    Nie tęsknię do kawiarni gwarnej\\
    gdzie mieszka banda dziwolągów\\
    gardzę zapachem buduaru\\
    gdzie Ania psoci wśród szezlongów

    Nie nęcą mnie zalety babin\\
    kobieta zdradną, bierz ją diabli\\
    kochanką moją jest karabin\\
    i klinga ukochanej szabli

    Niejeden wróg miał na mnie chrapkę\\
    a teraz jęczy w piekle na dnie\\
    ze śmiercią igram w ciuciubabkę\\
    więc może wkrótce mnie dopadnie

    Ksiądz nich mnie grzebie, albo rabin\\
    żołnierza się nie czepią diabli\\
    Lecz w grób połóżcie mi karabin\\
    i klingę ukochanej szabli

    (Inskrypcja na szabli malarza Henryka Szczyglińskiego, ułana 2 pułku Kawalerii Legionów Polskich wg tekstu  z 1899 r.)

\end{text}
\begin{chord}

\end{chord}