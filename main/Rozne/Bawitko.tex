%%
%% Author: bartek.rydz
%% 10.02.2019
%%
% Preamble
\tytul{Bawitko}{Andrzej Poniedzelski}{Andrzej Poniedzielski}
\begin{text}
    Wagą zabraną Temidzie\\
    Bawimy się w sprawiedliwość\\
    Na jednej szali zło kładziemy\\
    Na drugiej dobro i litość\\
    Na drugiej dobro i litość\\
    Wszyscy się cieszą z równowagi\\
    Gardła zdzieramy w wiwatach\\
    Wszyscy się cieszą z równowagi\\
    Wskazówkę puszczamy po latach\\
    Wskazówkę puszczamy po latach

    Oj nieładnie człowieku nieładnie\\
    Oj nieładnie człowieku brzydko\\
    Ty się całe życie bawisz\\
    Czasem sobie zmieniasz bawitko

    Księgami bawimy się w mądrość\\
    Zabawa to dla upartych\\
    Z ksiąg budujemy nauki i domy\\
    A przecież księgi to karty\\
    A przecież księgi to karty\\
    Raz huczą brawa raz działa\\
    Już się gubimy w erratach\\
    Na ile to mądre na ile starczy\\
    Ktoś nas osądzi po latach\\
    Ktoś nas osądzi po latach

    Jest jeszcze jedna zabawa\\
    Też popularna choć nie nowa\\
    Do niej potrzebnych jest dwoje ludzi\\
    I słowa i słowa i słowa\\
    I słowa i słowa i słowa\\
    Słowami bawimy się w miłość\\
    Słowa składamy w kwiatach\\
    Potem przyprószą je liście jesieni\\
    Odgrzebujemy po latach\\
    Odgrzebujemy po latach
\end{text}
\begin{chord}

\end{chord}