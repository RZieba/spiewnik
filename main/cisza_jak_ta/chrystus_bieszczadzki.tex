\tytul{Chrystus bieszczadzki}{sł. K. Patora, muz. M. Łangowski}{Cisza jak ta}
\begin{text}
\hfill\break
\hfill\break
Siedząc na swym pniaczku\\
jak bieszczadzki gazda\\
Błogosławisz ptakom\\
wracającym do gniazda\\
Tym, co przyszli tutaj bo z serca chcieli\\
I co wśród pożogi odejść stąd musieli

\vin Wskaż nam Panie drogę\\
\vin po Bieszczadzkich szlakach\\
\vin Zagubionym – bądź echem\\
\vin w strumieniach i ptakach\\
\vin I światłem w ciemności,\\
\vin jak twój księżyc blady\\
\vin Gdzie umilkły cerkwie\\
\vin i zdziczały sady

Tym, co przyszli tutaj by prawem zwyczaju\\
Podziękować Tobie za przedsionek raju\\
Za ptasie koncerty o porannym brzasku\\
Za lipcowe noce przy księżyca blasku\\

Wskaż nam Panie drogę\\
po Bieszczadzkich szlakach\\
Zagubionym – bądź echem\\
w strumieniach i ptakach\\
I światłem w ciemności,\\
jak twój księżyc blady\\
Gdzie umilkły cerkwie\\
i zdziczały sady

\hfill\break
Zieleń skryła blizny – zostały wspomnienia\\
W sercach został smak tamtego cierpienia\\
Znad tych samych ognisk inne pieśni płyną\\
Gnane ciepłym wiatrem\\
do wzgórz nad Soliną

Wskaż nam Panie drogę\\
po Bieszczadzkich szlakach\\
Zagubionym – bądź echem\\
w strumieniach i ptakach\\
I światłem w ciemności,\\
jak twój księżyc blady\\
Gdzie umilkły cerkwie\\
i zdziczały sady
\end{text}
\begin{chord}
D A G D\\
D A G D\\
Hm\\
A\\
G\\
D A\\
Hm A\\
G A

D\\
(Hm) A\\
G\\
(A) D\\
D\\
(Hm) A\\
G\\
(A) D

Hm A\\
G D A\\
Hm A\\
G A

D\\
(Hm) A\\
G\\
(A) D\\
D\\
(Hm) A\\
G\\
(A) D

C#m H A E H\\
C#m H A H\\
C#m H\\
A E H\\
C#m H\\
A H

E\\
(C#m) H\\
A\\
(H) E\\
E\\
(C#m) H\\
A\\
(H) E
\end{chord}
