\tytul{Chleba naszego poprzedniego}{sł. Wiesława Kwinto-Koczan, muz. M. Łangowski}{Cisza jak ta}
\begin{text}
\hfill\break
\hfill\break
A gdy zostaną mi tylko wspomnienia\\
I stopy po górach nie zechcą już nosić,\\
spojrzę do tyłu, zajrzę w głąb siebie\\
i będę Ciebie najgorliwiej prosić:

\hfill\break
Chleba naszego poprzedniego daj mi,\\
aby nakarmić głodnych myśli sforę,\\
żeby z tęsknoty przestały ujadać,\\
u kolan przycupnęły\\
w wieczorową porę.

Chleba naszego poprzedniego daj mi,\\
bo żyć nie umiem\\
w żaden inny sposób,\\
daj we wspomnieniach\\
jego zapach poczuć\\
pogodzić się łatwiej z zakrętami losu.

Chleba naszego poprzedniego daj mi\\
\end{text}
\begin{chord}
\textit{capo II}\\
C G a G a\\
a G a\\
C G Dm\\
a G Dm a\\
Dm Em F G a (G) a

a G a C\\
FGC (Em)a\\
FGC (Em)a\\
F G\\
a (G) a

a G a C\\
F G\\
C (Em) a\\
F G\\
C (Em) a\\
F G a (G) a

a G a (G) a
\end{chord}
