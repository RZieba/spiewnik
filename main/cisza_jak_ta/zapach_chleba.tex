\tytul{Zapach chleba}
{sł. i muz. M. Borowiec}
{Cisza jak ta}
\begin{text}
Kolejna noc Wielkim Wozem wędruje\\
Ze szczytu Tarnicy można go złapać za koło\\
Księżyc granie gór cienką kreską maluje\\
Na złoto, zielono i czerwono

\vin A góry się piętrzą i rosną do nieba\\
\vin Bieszczadzkie baśnie nam szepczą do ucha\\
\vin Zapachem połonin i pieczonego chleba\\
\vin Śnić się będą, gdy znowu zapanuje plucha.

Kolejny bar i stół pełen piwa\\
Na ławach zasiadły upadłe anioły\\
W kuflach skrzydlatych zmęczenie odpływa\\
Splatają się pieśni z dźwiękami gitary

\vin A góry się piętrzą i rosną do nieba\\
\vin Bieszczadzkie baśnie nam szepczą do ucha\\
\vin Zapachem połonin i pieczonego chleba\\
\vin Śnić się będą, gdy znowu zapanuje plucha.

\hfill\break
Kolejny szlak pnie się krętą drogą\\
Warkoczem połonin, panien roześmianych\\
Deptany ciężką, ludzką nogą\\
Prowadzi nas do schronisk – rajów obiecanych.

\vin A góry się piętrzą i rosną do nieba\\
\vin Bieszczadzkie baśnie nam szepczą do ucha\\
\vin Zapachem połonin i pieczonego chleba\\
\vin Śnić się będą, gdy znowu zapanuje plucha.
\end{text}
\begin{chord}
G C9/5 a7 D\\
G C9/5 a7 D\\
C D G e\\
C D G

C h e\\
C D G\\
C h e\\
C D e/G

G* C* a7 D\\
G* C* a7 D\\
C D G e\\
C D G

C h e\\
C D G\\
C h e\\
C D e/G\\
e a C h a h C D

G* C* a7 D\\
G* C* a7 D\\
C D G e\\
C D G

C h e\\
C D G\\
C h e\\
C D e/G
\end{chord}
