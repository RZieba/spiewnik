\tytul{Skowyt}
{sł. Wiesława Kwinto-Koczan, muz. Michał Łangowski}
{Cisza jak ta}
\begin{text}
\hfill\break
\hfill\break
\hfill\break
Kiedy słyszę ich melodię,\\
znowu jestem w wilczym chórze.\\
Razem z całym stadem wyję,\\
ciarki chodzą mi po skórze.

Czy to tylko zew natury,\\
czy też skowyt mojej duszy,\\
co ruszyła w białą pustkę,\\
żeby dźwiękiem ból zagłuszyć.

Szukam twoich śladów w śniegu,\\
w mgławe drzwi bezsilnie pukam\\
i nie wierzę, że cię nie ma,\\
chociaż wiem – na próżno szukam.

Ostrym tonem z ust wydartym,\\
sercu krwi upuszczę chętnie.\\
Niech już bólu się pozbędzie,\\
albo lepiej niech już pęknie.

\hfill\break
\hfill\break
\hfill\break
Ostrym tonem z ust wydartym,\\
sercu krwi upuszczę chętnie.\\
Niech już bólu się pozbędzie,\\
albo lepiej niech już pęknie.

Ostrym tonem z ust wydartym,\\
sercu krwi upuszczę chętnie.\\
Niech już bólu się pozbędzie,\\
albo lepiej niech już pęknie.\\
niech już pęknie\\
niech już pęknie\\
niech już pęknie
\end{text}
\begin{chord}
Fmaj9 (003010)\\
e11 (005030)\\
a2 (007500)\\
Fmaj9 e11\\
a2 Fmaj9\\
Fmaj9 e11\\
a2 Fmaj9

Fmaj9 e11\\
a2 Fmaj9\\
Fmaj9 e11\\
a2 Fmaj9

Fmaj9 e11\\
a2 Fmaj9\\
Fmaj9 e11\\
a2 Fmaj9

F G\\
a F\\
F G\\
a F\\
F G a F\\
Fmaj9 e11 / a2 Fmaj9

F G\\
a F\\
F G\\
a F

F G\\
a F\\
F G\\
a F\\
F\\
G\\
a
\end{chord}
