%%
%% Author: bartek.rydz
%% 19.02.2019
%%
% Preamble
\tytul{Wiosna 1905}{sł. J. Kaczmarski wg obrazu Masłowskiego, muz. Z. Łapiński 1980}{Jacek Kaczmarski}
\begin{text}
    \chordfill\\
    \chordfill\\
    Wiosenny dzień – zapach lip\\
    Po deszczu ulica lśni\\
    Stuk podków i siodeł skrzyp\\
    Koń wiosnę poczuł i rży\\
    Bandytów złapano dwóch\\
    Przez miasto prowadzą ich\\
    Drży koński przy twarzy brzuch\\
    Drży twarz przy końskim brzuchu

    Długo czekali na tę wiosnę\\
    Całe swoich piętnaście lat\\
    W marszu w Aleje Ujazdowskie\\
    Z twarzy uśmiech dziecięcy spadł\\
    Idą i drogi swej nie widzą\\
    W myślach pierwszy piszą już list\\
    Konie wędzidła swoje gryzą\\
    Uszy drąży kozacki gwizd

    \chordfill\\
    \chordfill\\
    To dzieci w słów wierzą sens\\
    To dzieci marzą i śnią\\
    To dzieciom sen spędza z rzęs\\
    Dobro płacone ich krwią\\
    Dorośli umieją żyć\\
    Dorosłym sen – mara – śmiech\\
    To dzieci będą się bić\\
    Za słów dorosłych prawdę

    Przeszli nie widać ich zza koni\\
    Znika po kałużach ich ślad\\
    Jeszcze w ostrogę szabla dzwoni\\
    Czymże byłby bez tego świat\\
    Pusta ulica pachnie deszczem\\
    W drzewach jasny puszy się liść\\
    W mokrym powietrzu ciągle jeszcze\\
    Krąży kozacki gwizd
\end{text}
\begin{chord}
    g G g G Gis A\\
    b B b B H C\\
    cis\\
    cis\\
    cis\\
    cis\\
    fis\\
    fis\\
    fis\\
    fis Fis7

    h D0\\
    h D0 (E7 A74-3)\\
    d D0\\
    d D0 (Fis7)\\
    h D0\\
    h D0 (E7 A74-3)\\
    d D0\\
    d D0 (B D4-3)\\
    g G g G Gis A\\
    b B b B H C

    cis\\
    cis\\
    cis\\
    cis\\
    fis\\
    fis\\
    fis\\
    fis Fis7

    h D0\\
    h D0 (E7 A74-3)\\
    d D0\\
    d D0 (Fis7)\\
    h D0\\
    h D0 (E7 A74-3)\\
    d D0\\
    d D0 (B D4-3)

    g G g G
\end{chord}