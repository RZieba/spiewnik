%%
%% Author: bartek.rydz
%% 1^{7}.02.201^{9}
%%
% Preamble
\tytul{Mury}{sł. Jacek Kaczmarski, muz. Llouis Lach 1980}{Jacek Kaczmarski}
\begin{text}
    On natchniony i młody był, ich nie policzyłby nikt\\
    On im dodawał pieśnią sił, śpiewał, że blisko już świt.\\
    Świec tysiące palili mu, znad głów podnosił się dym,\\
    Śpiewał, że czas, by runął mur...\\
    Oni śpiewali wraz z nim:

    Wyrwij murom zęby krat!\\
    Zerwij kajdany, połam bat!\\
    A mury runą, runą, runą\\
    I pogrzebią stary świat!

    Wyrwij murom zęby krat!\\
    Zerwij kajdany, połam bat!\\
    A mury runą, runą, runą\\
    I pogrzebią stary świat!

    Wkrótce na pamięć znali pieśń i sama melodia bez słów\\
    Niosła ze sobą starą treść, dreszcze na wskroś serc i głów.\\
    Śpiewali więc, klaskali w rytm, jak wystrzał poklask ich brzmiał,\\
    I ciążył łańcuch, zwlekał świt...\\
    On wciąż śpiewał i grał

    Wyrwij murom zęby krat!\\
    Zerwij kajdany, połam bat!\\
    A mury runą, runą, runą\\
    I pogrzebią stary świat!

    Wyrwij murom zęby krat!\\
    Zerwij kajdany, połam bat!\\
    A mury runą, runą, runą\\
    I pogrzebią stary świat!

    Aż zobaczyli ilu ich, poczuli siłę i czas,\\
    I z pieśnią, że już blisko świt szli ulicami miast;\\
    Zwalali pomniki i rwali bruk – Ten z nami! Ten przeciw nam!\\
    Kto sam – ten nasz najgorszy wróg!\\
    A śpiewak także był sam.

    Patrzył na równy tłumów marsz,\\
    Milczał wsłuchany w kroków huk,\\
    A mury rosły, rosły, rosły\\
    Łańcuch kołysał się u nóg...

    Patrzy na równy tłumów marsz,\\
    Milczy wsłuchany w kroków huk,\\
    A mury rosną, rosną, rosną\\
    Łańcuch kołysze się u nóg...
\end{text}
\begin{chord}
    e H^{7} e H^{7}\\
    C H^{7} C H^{7} e\\
    e H^{7} e H^{7}\\
    C H^{7} C\\
    C H^{7} e

    H^{7} e\\
    H^{7} e\\
    a e\\
    H^{7} e

    H^{7} e\\
    H^{7} e\\
    a e\\
    H^{7} e (e^{9} e e^{9} e e^{9})

    e H^{7} e H^{7}\\
    C H^{7} C H^{7} e (e^{9} e e^{9})\\
    e H^{7} e H^{7}\\
    C H^{7} C\\
    C H^{7} e

    H^{7} e\\
    H^{7} e\\
    a e\\
    H^{7} e

    \hfill\break
    \hfill\break
    \hfill\break
    (e^{9} e e^{9} e e^{9})

    e H^{7} e H^{7}\\
    C H^{7} C H^{7} e (e^{9} e e^{9})\\
    e H^{7} e H^{7}\\
    C H^{7} C\\
    C H^{7} e

    H^{7} e\\
    H^{7} e\\
    a e\\
    H^{7} e

    H^{7} e\\
    H^{7} e\\
    a e\\
    H^{7} e

    H^{7} e H^{7} e a e H^{7} e
\end{chord}