%%\\
%% Author: bartek.rydz\\
%% 17.02.2019\\
%%\\
% Preamble\\
\tytul{Pompeje}{sł. i muz. Jacek Kaczmarski 1979}{Jacek Kaczmarski}
\begin{text}
    Odkopywaliśmy miasto Pompeję\\
    Jak się odkrywa spodziewane lądy\\
    Gdy okiem ludzkim nie widziane dzieje\\
    Jutro ujrzane potwierdzą poglądy\\
    Z których dziś jeszcze głupców tłum się śmieje

    W miarę kopania miejski cień narastał\\
    Jakbyśmy wszyscy wracali do domu\\
    Wjeżdżając wolno w świt wielkiego miasta\\
    Cicho, by snu nie przerywać nikomu\\
    Tylko pies szczekał i łańcuchem szastał

    - Czemu pies szczeka, rwie się na łańcuchu?\\
    Szybkie dłonie masują mięśnie dostojnika\\
    Który w łaźni na własnym kołysząc się brzuchu\\
    Wydaje rozkazy, pyta niewolnika\\
    - Pies szczeka, bo się boi wielkiego wybuchu!

    Bzdura. Na chwilę przerwij, bolą kości\\
    Co z poetą, którego rozkazałem śledzić?\\
    - Nic nowego meldują podwładni z ciemności\\
    W swoim mieszkaniu przy kaganku siedzi\\
    I pisze za wierszem wiersz dla potomności

    - Czemu pies szczeka, targa się po nocy?\\
    Tej, którą objął twarz znieruchomiała\\
    - Bzdura, może chuligan trafił kundla z procy\\
    Mruczy, chce wydobyć uległość z jej ciała\\
    Ale ona w oknie utkwiła już oczy

    - Ziemia drży czy nie czujesz? Objął ją od tyłu\\
    I szepnął do ucha - to drżą członki moje!\\
    Świat nie zginie dlatego, że bydle zawyło\\
    Odwraca jej głowę i długo całuje\\
    Na dach pada gorące pierwsze ziarno pyłu

    - Czemu pies szczeka, słychać w całym mieście?\\
    Więzień szarpie kratę, czuje duszny powiew\\
    - Strażnicy otwórzcie! Ludzie, gdzie jesteście?\\
    - Ja jestem - żebrak spod muru odpowie\\
    Ślini się, nóg nie ma, drzemał przy areszcie

    - Ratuj mnie wypuść! - ten tylko się ślini\\
    I ślina w ciemnościach już błyszczy czerwono\\
    - Mogę pomodlić się w jakiejś świątyni\\
    Tyle ich ostatnio tutaj postawiono\\
    Nic żaden z bogów dla nas nie uczyni!

    - Czemu pies szczeka, patrz jak płonie niebo\\
    Z pieca wyciągaj bochenki niezdaro\\
    Ziemia dygoce uciekać stąd trzeba\\
    Nie chcę być własnej głupoty ofiarą!\\
    Weź wszystkie pieniądze i formę do chleba!

    - Czemu pies szczeka?! Tak to już koniec\\
    Lecz jeszcze zabiorę te misy z ołtarza\\
    Nikt nie zobaczy gdy wszystko zniszczone!\\
    Nie zdążę, nie zdążę! Noc w dzień się rozjarza\\
    Biegnę, jak ciężko! Powietrze spalone...

    Psa, który ostrzegał, nikt nie spuścił z łańcucha\\
    Zastygł, pysk otwarty, łapy w próg wtopione

    Podniosłem oczy i objąłem wzrokiem\\
    Ulice, stragany, stadiony sklepienia\\
    Wtem słyszę szmer\\
    Pada z nieba popiół...\\
    A to się tylko obsunęła ziemia\\
    Pod czyimś szybkim nierozważnym krokiem
\end{text}
\begin{chord}
    e\\
    A\\
    a\\
    e\\
    a H^{7} e

    G\\
    D\\
    a\\
    e\\
    a H^{7} e^{9}

    e\\
    A\\
    a\\
    e\\
    a H^{7} e

    G\\
    D\\
    a\\
    e\\
    a H^{7} e^{9}

    e\\
    A\\
    a\\
    e\\
    a H^{7} e

    G\\
    D\\
    a\\
    e\\
    a H^{7} e^{9}

    e\\
    A\\
    a\\
    e\\
    a H^{7} e

    G\\
    D\\
    a\\
    e\\
    a H^{7} e^{9}

    e\\
    A\\
    a\\
    e\\
    a H^{7} e

    G\\
    D\\
    a\\
    e\\
    a H^{7} e^{9}

    a\\
    h

    e\\
    G\\
    H\\
    a\\
    a e\\
    a H^{7} e^{9}
\end{chord}