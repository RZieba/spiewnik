%%
%% Author: bartosz.rydz
%% %28.05.2018
%%
\tytul{Życie osła}{sł. D. Ślepowrońska, muz. S. Brzozowski}{Czerwony Tulipan}
\begin{text}
    Czy życie osła musi być zawsze ośle?\\
    Czy osioł żyć nie może anielsko, wzniośle?\\
    Ochrypłym, drżącym głosem, a nawet z łezką w oku\\
    przemawiał pan osiołek do łąki i obłoków\\
    hej, ho do łąki i obłoków,\\
    hej ho do obłoków

    Ech, gdyby tak skrzydła mieć ogromnie i tęczowe\\
    złotą aureolą otulić oślą głowę\\
    Gdyby tak skrzydła mieć ogromnie i tęczowe\\
    złotą aureolą otulić oślą głowę,\\
    hej, ho gdyby tak skrzydła mieć,\\
    hej, ho gdyby tak skrzydła mieć

    Był dobry, bardzo dobry, na bąki nie śmiał gderać.\\
    Lecz czyż kto żył jak osioł, jak osioł ma umierać?\\
    Może kiedy łagodnie bez złości przejdą lata\\
    rzuci grosz mu świętości tata z tamtego świata\\
    hej, ho tata z tamtego świata,\\
    hej, ho z tamtego świata.

    Już koniec, jak żył, tak umarł nasz osiołek\\
    dębowe spuśćmy wieko, przykryjmy ziemią dołek.\\
    Przy ostatniej skipce zanućmy od niechcenia,\\
    że może jednak czasem spełniają się marzenia\\
    hej, ho spełniają się marzenia,\\
    hej, ho marzenia.
\end{text}
\begin{chord}
    d G d C G\\
    d G d C G\\
    C G F C\\
    dCFdFCdd\\
    G d C d\\
    G d C d

    A C F C\\
    d C F d C d\\
    A C F C\\
    d C F d C d\\
    G d C d\\
    G d C d\\
    (d G d C G)\\
    d G d C G\\
    d G d C G\\
    C G F C\\
    dCFdFCdd\\
    G d C d\\
    G d C d

    d G d C G\\
    d G d C G\\
    C G F C\\
    dCFdFCdd\\
    G d C d\\
    G d C d
\end{chord}