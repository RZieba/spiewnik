%%
%% Author: bartek.rydz
%% 28.05.2018
%%
% Preamble
\tytul{Teatr uliczny}{}{Czerwony Tulipan}
\begin{text}
    Teatr uliczny, teatr to śliczny\\
    prawdziwych przeżyć pełna gama\\
    Jedni się śmieją inni smucą\\
    Podział na role – dramat\\
    Miejsce na placu przy fontannie\\
    Albo na skwerze obok Łyny\\
    Na każdym rogu w każdej bramie\\
    Maska za maską suną – mimy

    Raz przyszedł taki ktoś kipiący mądrością\\
    Powiedział mi że czas skończyć z tą młodością\\
    Powiedział że czas żyć bardziej rozumnie\\
    Bo nadszedł już mój życia średni wiek\\
    Bo nadszedł już mój życia średni wiek\\
    A ja w naiwnej radości\\
    Nie chcę rezygnować ze swojej młodości

    Teatr uliczny, teatr to śliczny\\
    Podział na aktów ciasną przestrzeń\\
    Cichutko idą arlekiny\\
    Po białych buziach toczą łezkę\\
    Za nimi klauni wesołkowie\\
    Śmieszni do bólu na tle tłumu\\
    Tym z twarzy tryska samo zdrowie\\
    Za cieniem chowają się kostiumu

    Przyszedł taki ktoś przelany mądrością\\
    Powiedział mi że czas skończyć z tą miłością\\
    Powiedział że czas żyć bardziej rozumnie\\
    Bo nadszedł już mój życia średni wiek\\
    Bo nadszedł już mój życia średni wiek\\
    A ja w naiwnej radości\\
    Nie chcę rezygnować ze swojej miłości

    Teatr uliczny, teatr to śliczny\\
    Koniec przychodzi w masce sprawy\\
    Nikną kuglarze arlekiny\\
    Zostaje scena żądna sławy\\
    Porzucam pozy zbędne ruchy\\
    A z twarzy zrywam lekki grymas\\
    W uszach mi jeszcze grają duchy\\
    Opada ciszy peleryna

    Aż przyszedł zwykły ktoś\\
    Zwyczajną mądrością\\
    Powiedział – ty się ciesz\\
    Że żyjesz miłością\\
    Powiedział – dobrze jest\\
    Że radość jest z Tobą\\
    Bo piękny jest twój życia średni wiek\\
    Bo piękny jest twój życia średni wiek

    Teatr uliczny, teatr to śliczny\\
    Prawdziwych przeżyć pełna gama\\
    Jedni się śmieją inni smucą\\
    Podział na role – dramat
\end{text}
\begin{chord}

\end{chord}