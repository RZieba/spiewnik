%%
%% Author: bartek.rydz
%% 14.02.2019
%%
% Preamble
\tytul{Wiatraki}{sł. Władysław Broniewski, muz. Tomek Fojgt}{Enigma}
\begin{text}
    \chordfill
    W wiatry wplątane czarne ich ręce\\
    Chmury wełniste szarpią i drą\\
    Kręcą, kołują, skrzypią i kręcą\\
    Głuszą krakanie kruków i wron.

    Ramię okropne macha i macha\\
    Piersiom drewnianym brakuje już tchu\\
    Wierzby kudłate pędzą w przestrachu\\
    Bledną chałupy w czapach ze mchu.

    Kręcą, kołują, wiercą, chrobocą\\
    Wre czarnoksięski trzepot ich rąk\\
    Drogę torują wiatrom i nocom\\
    Włóczą za włosy mgły sponad łąk.

    Wory obłoków sypią im w żarna\\
    Rudych wieczorów siarkę i miedź\\
    Miele się mąka gęsta i czarna\\
    W niebo gwieździste spada jak w sieć.

    Słońce językiem strugę krwi liże\\
    W pianie zachodu tarza się dzień\\
    Wtedy ich ręce sterczą jak krzyże\\
    Widzę rozpięty na nich mój cień.

    Dławią go, szarpią, męczą od rana\\
    Czarne ramiona ostre jak krzyk\\
    W niebo krwawiące świtem jak rana\\
    Cień oszalały skoczył i znikł.
\end{text}
\begin{chord}
    \textit{Capo I}\\
    h A\\
    G e fis G\\
    h A\\
    G e fis G

    h D\\
    e G\\
    h D\\
    e G

    D E\\
    G\\
    h E\\
    D A fis h\\
    H
\end{chord}