%%
%% Author: bartek.rydz
%% 14.02.2019
%%
% Preamble
\tytul{Ulewa}{sł. Adam Asnyk, muz. Tomek Fojgt}{Enigma}
\begin{text}
    Na piętra gór, na ciemny bór\\
    Zasłony spadły sine\\
    W deszczowych łzach granitów gmach\\
    Rozpłynął się w równinę\\
    Nie widać nic - błękitów tło\\
    I całe widnokręgi\\
    Zasnute w cień, zalane mgłą\\
    Porżnięte w deszczu pręgi

    Na szczytach Tatr, na szczytach Tatr\\
    Na sinej ich krawędzi\\
    Króluje w mgłach świszczący wiatr\\
    I ciemne chmury pędzi\\
    Rozpostarł z mgły utkany płaszcz\\
    I rosę z chmur wyciska\\
    A szczyty wód z wilgotnych paszcz\\
    Spływają na urwiska

    I dzień, i noc, i nowy wschód\\
    Przechodzą bez odmiany\\
    Dokoła szum rosnących wód\\
    Strop niebios ołowiany\\
    I siecze deszcz, i świszcze wiatr\\
    I głośniej się znów potok gniewa\\
    Na szczytach Tatr, w dolinach Tatr\\
    Mrok szary i ulewa
\end{text}
\begin{chord}
    A A7+\\
    E fis D\\
    A A7+\\
    E fis\\
    h A G fis E\\
    h A G fis E\\
    A E fis D\\
    h E4 E A

    D cis fis\\
    D cis fis\\
    h A G fis E\\
    D E Fis\\
    D cis fis\\
    D cis fis\\
    h A G fis E\\
    D E Fis
\end{chord}