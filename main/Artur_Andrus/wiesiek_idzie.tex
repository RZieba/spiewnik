\tytul{Wiesiek idzie}\\
{sł. A. Andrus, muz. A. Grotowski}\\
{Artur Andrus}\\
\begin{text}\\
Raz staruszka błądzącego w lesie.\\
Ujrzał lisek przywiędły i blady.\\
I pomyślał: 'Znowu idzie Wiesiek\\
Wiesiek idzie, nie ma na to rady'

I podreptał do nory po ścieżce\\
I oznajmił stanąwszy przed chatą.\\
Swojej żonie lisicy Agnieszce\\
Wiesiek idzie, nie ma rady na to.

Zaś lisica zmartwiła się szczerze\\
I machnęła łapkami obiema\\
Matko Boska! Bądź ostrożny, Jerzy\\
Wiesiek idzie, rady na to nie ma

Może przybyć już dziś albo jutro\\
Lub pojutrze, a może za tydzień\\
Może nieźle przetrzepać nam futro\\
Nie ma rady, Wiesiek, Wiesiek idzie

A był sierpień, pogoda prześliczna\\
I tętniło życie w zagajnikach.\\
Oprócz lisów nikt chyba nie myślał\\
O nadejściu Wieśka kłusownika

Ale cóż, one żyły dość długo\\
Łby na karkach miały nie od parady.\\
I wiedziały, że prędzej czy później\\
Wiesiek przyjdzie, nie ma na to rady
\end{text}
\begin{chord}
\end{chord}