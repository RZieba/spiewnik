%%
%% Author: bartek.rydz
%% 19.02.2019
%%
% Preamble
\tytul{Hymn}{sł. J. Kaczmarski, muz. P. Gintrowski 1980}{Jacek Kaczmarski}
\begin{text}
    Kto w twierdzy wyrósł po co mu ogrody\\
    Kto krew ma w oczach nie zniesie błękitu\\
    Sytość zabije nawykłych do głodu\\
    Myśl upodlona nie dźwignie zaszczytów

    Kto się ukrywał szczuty i tropiony\\
    Nigdy nie będzie umiał stanąć prosto\\
    Zawsze pod murem zawsze pochylony\\
    Chyba że nagle uwierzy w swą boskość

    Kto w życiu oparł się wszelkim pokusom\\
    Tu po tygodniu jest ostoją grzechu\\
    Dawniej jedyną uciśnioną duszą\\
    A teraz jednym z tysięcy uśmiechów

    Z ran zadawanych kto krzyczał w agonii\\
    Na męki dając ciało by myśl pieścić\\
    Nie widzi sensu w nauce harmonii\\
    Ani nie umie śpiewać o tym pieśni

    Kto świat opuścił w zastanej postaci\\
    Ten nie zaśpiewa hymnu dziękczynienia\\
    Choćby i przez to miał zbawienie stracić\\
    Nie ma o nie ma zadośćuczynienia\\
    Nie ma o nie ma zadośćuczynienia\\
    Nie ma o nie ma zadośćuczynienia

    Nie dla wiwatów człowiekowi dłonie\\
    Nie dla nagrody przykazania boże\\
    Bo każdy oddech w walce ma swój koniec\\
    Walka o duszę\\
    Końca mieć nie może
\end{text}
\begin{chord}
    a E\\
    a E\\
    F e C a e a E\\
    F e C a e a E

    a E\\
    a E\\
    F e C a e a E\\
    F e C a e a E

    e h\\
    e h\\
    C D e\\
    C D e

    e h\\
    e h\\
    C D e\\
    C D e

    e h\\
    e h\\
    C D e\\
    C D e\\
    C D e\\
    C D e

    a E\\
    a E\\
    F e C a e a E\\
    F e\\
    C a e a E
\end{chord}