%%
%% Author: bartek.rydz
%% 17.02.2019
%%
% Preamble
\tytul{Pejzaż z szubienicą}{sł. Jacek Kaczmarski, muz. Przemysław Gintrowski  1978}{Jacek Kaczmarski}
\begin{text}
    - Dokąd prowadzicie, rozpaleni wesołkowie?\\
    Rumiane baby, chłopi pijani w siwy dym?

    Na górę prowadzimy Cię przez leśne pustkowie,\\
    Na góry wyłysiały, dziki szczyt.

    - Za szybko prowadzicie, tańczycie upojeni,\\
    Już nie mam sił, by w waszym korowodzie iść!

    Nie przejmuj się człowieku, na rękach poniesiemy!\\
    Nie zostawimy Cię, Tyś Królem dziś!\\
    Nie zostawimy Cię, Tyś Królem dziś!

    - Szczyt widać poprzez drzewa, cel waszej drogi bliski...\\
    Stać! Puśćcie mnie na ziemię! Co tam na szczycie tkwi?

    - To tron na Ciebie czeka! Ramiona nasze niskie,\\
    Panować stamtąd łatwiej będzie Ci!

    - Ja nie chce takiej władzy, podstępni hołdownicy!\\
    Wasz taniec moją śmiercią, milczeniem mym wasz chór!

    - Zatańczysz, zapanujesz nam na tej szubienicy!\\
    Zaskrzypi nam do taktu ciężki sznur!

    No, jak Ci tam na górze? Gdzie tak wytrzeszczasz gały?\\
    Dlaczego pokazujesz złośliwy jęzor nam?\\
    Nie tańczysz jak należy, kołyszesz się nieśmiało!\\
    Cóż ciekawego zobaczyłeś tam?!

    - Tu jasne są przestrzenie i widzę krągłość ziemi,\\
    Gdy czasem wiatr podrzuci mnie ponad czarny las,\\
    Bez lęku tańczcie dalej w gęstwinie własnych cieni,\\
    I tak nikt nigdy nie zobaczy was...

    Tu jasne są przestrzenie i widzę krągłość ziemi,\\
    Gdy czasem wiatr podrzuci mnie ponad czarny las,\\
    Bez lęku tańczcie dalej w gęstwinie własnych cieni,\\
    I tak nikt nigdy nie zobaczy was...
\end{text}
\begin{chord}
    D e A D\\
    H e A h (A)

    D e A D\\
    H e A h (A)

    D c D c\\
    F B F B

    g Dis D g Dis D\\
    g f\\
    g f

    A D A\\
    c B f (a)

    F a F a\\
    F a (F a)

    e (h)\\
    e (h)

    h\\
    h Fis h

    c\\
    c G c\\
    d\\
    d A d

    e a e a\\
    D e\\
    e a e a\\
    D e

    e a e a\\
    D e\\
    e a e a\\
    D e
\end{chord}