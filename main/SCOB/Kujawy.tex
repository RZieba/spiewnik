%%
%% Author: bartosz.rydz
%% %29.05.2018
%%
\tytul{Kujawy}{}{Słodki Całus od Buby}
\begin{text}
    Spotkałem go wtedy na stacji przesiadek,\\
    Gdzie minut czterdzieści i w drogę znów wio,\\
    Gdzie pociąg z Brodnicy albo na Włocławek\\
    I nie wiesz na pewno, gdzie rzuci cię los.

    Gdy pociąg kolejnych podróżnych wypluwał,\\
    W czeluści tunelu połykał ich mrok,\\
    Siedziałem na ławce, czas wolno upływał,\\
    Gdy nagle wzrok jego napotkał mój wzrok.

    \vin Chodź, stary, na piwo, a może i wódkę,\\
    \vin Wypijmy za życie, za niefart i fart,\\
    \vin Za tamtą dziewczynę spotkaną pod Kutnem,\\
    \vin Musiałeś wyjechać, a teraz ci żal.

    O naszych przygodach lud śpiewał już pieśni:\\
    Włóczęga, muzyka i życie jak dzwon...\\
    I nagle w tunelu na stacji przesiadek\\
    Rozbłysły wspomnienia w uścisku dwóch rąk.

    Chodź, stary, na piwo...

    Więc, stary, znów razem i świat znów przed nami...\\
    Po chwili od złudzeń aż mętniał nam wzrok,\\
    Lecz czas się nie cofnie na stacji przesiadek,\\
    Zostają wspomnienia, cóż, dobre i to.
\end{text}
\begin{chord}

\end{chord}