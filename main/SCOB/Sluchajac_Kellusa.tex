\tytul{Słuchając Kellusa}{sł. muz. Krzysztof Jurkiewicz}{Słodki Całus od Buby}
\begin{text}
Gdy z mojego domu zechcesz wyjrzeć oknem,\\
Z jednej strony wzgórza, z drugiej masz Zatokę,\\
Z jednej strony wzgórza, z drugiej masz Zatokę,\\
Białą wstęgę Helu gdzieś na horyzoncie. 

Ponad głową chmury jak nigdzie na świecie,\\ 
Wiatr od morza słony oddech nocy niesie,\\
Słony oddech nocy, białe skrzydła mewy, \\
Kilka słów poezji i pijackie śpiewy.

W lekkiej mgiełce strachu błądzą zakochani,\\ 
W parku, który nigdy nie miał dobrej sławy, \\
W parku, który nigdy nie miał dobrej sławy,\\
Ci biją dla forsy, tamci- dla zabawy. 

W betonowej wieży rzędy brudnych okien, \\
Prawie co dzień życie ktoś zakańcza skokiem,\\ 
Prawie co dzień życie ktoś zakańcza skokiem,\\
Potem cicho leży na jezdni pod blokiem. 

Chłodna, Jasna, Polna- wszystkie takie same,\\
Każdy dom wyszczerza nagich cegieł plamę,\\
Każdy dom wyszczerza nagich cegieł plamę,\\
Jak Bermudzki Trójkąt bramy zakazane. 

Czasem sobie myślę, słuchając Kellusa,\\
Czasem nic nie myślę, gdy Kellusa słucham,\\ 
Czasem ludzkie sprawy, czasem wielkie filozofie,\\ 
Czasem tylko jeże łapię gdzieś na szosie...
\end{text}
\begin{chord}
    G C G\\
    G C G\\
    D e G\\
    D C G

G C G\\
G C G\\
D e G\\
D C G

G C G\\
G C G\\
D e G\\
D C G

G C G\\
G C G\\
D e G\\
D C G

G C G\\
G C G\\
D e G\\
D C G

G C G\\
G C G\\
D e G\\
D C G
\end{chord}