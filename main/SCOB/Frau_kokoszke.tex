%%
%% Author: bartek.rydz
%% 07.02.2019
%%
% Preamble
\tytul{Frau Kokoschke}{sł. muz. Krzysztof Jurkiewicz}{Gdańska Formacja Szantowa}
\begin{text}
    Frau Kokoschke w oknie śpi,\\
    Po niemiecku liczy dni:\\
    Przecież jak dopłynie, zaraz miał dać znać…\\
    Wilhelm Gustloff - lieber Gott,\\
    To już sześćdziesiąt lat…

    \vin Mieć nadzieję i czekać,\\
    \vin Z marzeniami zwlekać:\\
    \vin Może wrócić dziś lub jutro,\\
    \vin Za lat dziesięć, może już za rok…\\
    \vin Przecież cuda się zdarzają wciąż…\\
    \vin Przecież cuda się zdarzają wciąż…\\
    \vin Cuda się zdarzają wciąż…

    Sama z dziećmi dziesięć lat -\\
    Wtedy znaleziono wrak,\\
    Napisali „zaginiony”, a on taki młody,\\
    tak potrzebny był…\\
    Ani żyje ani umarł, czasem brak już sił…

    Katastrofa - kilka chwil.\\
    Dobrze wie - nie przeżył nikt,\\
    Ale wciąż nie może się pogodzić, wciąż nadzieję ma.\\
    To prasuje mu koszule,\\
    To do morza wrzuci kwiat…

    Frau Kokoschke w oknie śpi,\\
    Biała suknia jej się śni,\\
    Dolne Miasto w kwiatach, ukochany Hans…\\
    Przecież cuda się zdarzają…\\
    Cuda ciągle się zdarzają…\\
    Cuda ciągle się zdarzają…\\
    Siwe włosy głaszcze znad Zatoki wiatr…
\end{text}
\begin{chord}
    h G\\
    h G\\
    a^7 D G\\
    e C\\
    D G

    e D C\\
    e D C\\
    a^7\\
    D G\\
    C D e\\
    C D e\\
    C D G
\end{chord}