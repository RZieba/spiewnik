Zwiewność
sł. B. Leśmian, muz. Z. Stefański
/ Am G F E7 (2x)
Brzęk muchy w pustym dzbanie, co stoi na półce / Am
Smuga w oczach po znikłej za oknem jaskółce / G
Cień ręki na murawie, a wszystko niczyje / F
Ledwo się zazieleni, już ufa że żyje / E E7

A jak dumnie się modrzy u ciszy podnóża / Am
Jak buńczucznie do boju z mgłą się napurpurza / G
A jest go tak niewiele, że mniej niż niebiesko / F
Nic prócz tła, biały obłok z czerwoną przekreską / E E7

Dal świata w ślepiach wróbla, spotkanie traw z ciałem / Am
Szmery w studni, ja w lesie, byłeś mgłą - bywałem / G
Usta twoje w alei, świt pod groblą, w młynie / F
Słońce w bramie na oścież, zgon pszczół w koniczynie / E E7

A jak dumnie się modrzy u ciszy podnóża / Am
Jak buńczucznie do boju z mgłą się napurpurza / G
A jest go tak niewiele, że mniej niż niebiesko / F
Nic prócz tła, biały obłok z czerwoną przekreską / E E7

Chód po ziemi człowieka, co na widnokresie / Am
Malejąc mało zwiewną gęstwę ciała niesie / G
I w tej gęstwie się modli i gmatwa co chwila / F
I wyziera z gęstwy w świat i na motyla / E E7

A jak dumnie się modrzy u ciszy podnóża / Am
Jak buńczucznie do boju z mgłą się napurpurza / G
A jest go tak niewiele, że mniej niż niebiesko / F
Nic prócz tła, biały obłok z czerwoną przekreską / E E7