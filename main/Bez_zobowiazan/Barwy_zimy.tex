\tytul{Barwy zimy}{sł. R. Marcinkowski, muz. G. Śmiałowski}{Bez Zobowiązań}
\begin{text}
Czerń: pośród śniegów - czeluście łąk.\\
Pod słońce. Wiecznie pod słońce,\\
w jasność objawień,\\
w oślepienie...\\
Z niezdarnego lotu\\
co chwila spadam w pejzaż\\
pełen niewiary.\\
I niemiłosiernie razi lustro rzeki...

Srebro: woda strumieni.\\
Jeszcze dzień. Wiecznie za krótki,\\
umyka rzeka...\\
Niedosyt: niedostatek uwagi.\\
Dosięgnąć głazów, opoki,\\
z g ł ę b i ć\\
i runąć - wzwyż,\\
w lodowatą toń powietrza!

I biel lodu: płaszczyzna abstrakcji.\\
Zanik wątpliwości. Tafla.\\
Doskonale nieludzka,\\
równa śmierci.\\
Wiele wyniknie jeszcze\\
z bólu roztopów.\\
Lecz teraz - nicość.\\
Abstrakcyjna płaszczyzna.
\end{text}
\begin{chord}
\end{chord}