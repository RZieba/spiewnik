\section{Czekaj mnie}
\begin{text}
Czekaj mnie, a wrócę zdrów,\\
Tylko czekaj mnie,\\
Gdy przekwitną kiście bzów,\\
Gdy naprószy śnieg.\\
Czekaj, gdy kominek zgasł,\\
Żar w popiele znikł.\\
Czekaj, gdy nikogo z nas\\
Już nie czeka nikt,\\
Czekaj, gdy po przejściu burz\\
Nie nadchodzi wieść,\\
Czekaj, gdy czekania już\\
Niepodobna znieść.

Czekaj mnie, a wrócę zdrów,\\
I nie pytaj gwiazd,\\
I nie słuchaj trzeźwych słów,\\
Że zapomnieć czas.\\
Niech opłacze matka, syn,\\
Gdy zaginie słuch,\\
Gorzkie wino w domu mym\\
Niech rozleje druh.\\
Za mój cichy, wieczny sen\\
kielich pójdzie w krąg.\\
Czekaj – i po kielich ten\\
Nie wyciągaj rąk.

Czekaj mnie, a wrócę zdrów,\\
Śmierci mej na złość.\\
Ten zaklaszcze, tamten znów\\
Krzyknie: „Co za gość!”.\\
Jak doprawdy pojąć im,\\
Że we krwawej mgle\\
Ty czekaniem cichym swym\\
Ocaliłaś mnie.\\
Ot i sekret, ot i znak,\\
Co w sekrecie tkwi,\\
Że umiałaś czekać tak,\\
Jak nie czekał nikt.
\end{text}