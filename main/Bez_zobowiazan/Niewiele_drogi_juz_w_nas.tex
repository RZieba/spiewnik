%%
%% Author: bartek.rydz
%% 07.02.2019
%%
% Preamble
\tytul{Niewiele drogi już w nas}{sł. muz. Robert Marcinkowski}{Robert Marcinkowski}
\begin{text}
    Niewiele drogi już w nas,\\
    gdy niebo się chyli i kończy się dzień.\\
    Niewiele drogi od marzeń\\
    do chusty pejzażu, co przywiera do stóp.

    Plączą się wstęgi i drogi\\
    zmieniają się w krew.\\
    Zrudziałe słońca lat spływają w dół.\\
    W studni pamięci twarze toną.

    Niewiele zmienia w nas pamięć,\\
    choć skrzy się jak diament i łasi co noc.\\
    Świętości głodni wracamy\\
    pod strzechy miraży, pod dziurawy dach.

    Z dziurawej nocy\\
    sypią się gwiazdy, piach dni...\\
    Sypią się zamki i wydmy, stargana myśl\\
    krąży z krukami nad równiną.

    Tam, gdzie nas nie ma, gdzie dobry świat –\\
    chyli się Ziemia we mgłach.\\
    Tam, skąd sfrunęliśmy, powietrze jeszcze lśni –\\
    w krainie wczoraj. W czyimś dziś.

    Niewiele drogi już w nas,\\
    gdy w piersi klepsydrze przesypał się krzyk.\\
    Świętości głodni wracamy\\
    pod strzechy miraży, pod dziurawy dach.

    Zrudziałe wstęgi pejzaży\\
    przywarły do stóp.\\
    Pogasły słońca lat.\\
    W klepsydrze serca milczy echo.
\end{text}
\begin{chord}

\end{chord}