\section{Miasteczko w Karpatach}
\begin{text}
Na jednym końcu Jan Chrzciciel chudziutki nie chce łakomstwem zgrzeszyć\\
Więc tłuste kury z kaplicy wygania aż trzepoczą nad miasteczkiem\\
Naprzeciw Święty Florian pod nosa czerwonym żaglem\\
Ma jak półksiężyc złotą łódkę wąsów i od ognia tulipanów jest\\
Odgrodzony płoteczkiem

Uliczki kot przeskakuje z jednej rynny na drugą\\
Maleńkiemu świątkowi ciernie ptaki wyciągają z pięty\\
I fura ze zbożem bez koni z dyszlem jak laską na ramieniu\\
Jak dziad wygląda kudłaty a obok pies leży w kurzu wyciągnięty

Na szubienicy daszku piekarniczego wozu co jedzie aby aby\\
Woźnica schylił głowę powieszony na drzemce\\
I ku brązowej kopułce cerkiewki myśląc że to orzech\
Chmury zachodów rude wiewiórzyce zza jodeł się wychylają poruszone wielce

A gdy nocą burza o byle świerk opiera błyskawic srebrny kaganiec\\
Wtedy na ciemne szumiące deszczem góry otwieramy okno\\
I śni się nam niedźwiedź co objąwszy konia białogrzywego za szyję\\
Wolno puszczą słoneczną jedzie i jak dziecko śni słodko
\end{text}