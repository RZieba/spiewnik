\section{Punkty na mapie}
\begin{text}
Lato w góry podeszło leniwie\\
Upał lisim zakradał się krokiem\\
Pamiętam, jak w rozpostarte ramiona\\
Światło strumieniem się lało z wysoka

Z wyschniętej studni nie czerpałem chłodu\\
O którym z rzadka wiatr opowiadał\\
Słony smak w morzu gór miała woda\\
A jednak chciało się zostać w Bieszczadach

Wiatr przywiał jesień w chmurze liści\\
Pogasły z wolna na drzewach zielenie\\
Pamiętam, jak w górach tak uroczystych\\
Las w coraz dłuższe ubierał się cienie

I ciepła tyle, co w kolorach drzewa\\
Minęły wieczory późne i gorące\\
Spadała ciężka już kurtyna nieba\\
Lecz z cięższym sercem porzucałem Gorce

Łączę palcami dwa punkty na mapie\\
Tak blisko, a przecież drogi tyle\\
Drogi z beztroski pełnym plecakiem\\
Nad której końcem żal się pochylał

I znów mi się marzy ta cisza święta\\
Gdzieś w zapomnianym przez ludzi schronisku\\
Bo tam resztę świata się ledwie pamięta\\
A to, co we mnie, to wszystko

Zima po śniegu nadeszła bezgłośnie\\
Białym u wierchów otulona futrem\\
Pamiętam okna zawiane po szyję\\
I czyjeś myśli zasypane smutkiem

Szlak się w śnieżycy całkiem pogmatwał\\
Co było blisko, stało się dalekie\\
Wlokła się z bólem wędrówka do światła\\
Choć bardziej bolało pożegnać Sudety

Wezbranym potokiem przypłynęła wiosna\\
Po szczytach pierwsze przetoczyła grzmoty\\
Pamiętam słońca zmagania ze śniegiem\\
I spoza chmur zwycięskie powroty

Chociaż marzyło się nieraz jeszcze\\
O suchym kącie w pasterskiej kolibie\\
Przed rozkapryszonym kryjąc się deszczem\\
To jakże bym mógł opuścić Beskidy?

Łączę palcami dwa punkty na mapie\\
Tak blisko, a przecież drogi tyle\\
Drogi z beztroski pełnym plecakiem\\
Nad której końcem żal się pochylał

I znów mi się marzy ta cisza święta\\
Gdzieś w zapomnianym przez ludzi schronisku\\
Bo tam resztę świata się ledwie pamięta\\
A to, co we mnie, to wszystko
\end{text}