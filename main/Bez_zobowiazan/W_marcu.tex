\section{W marcu nad ranem}
\begin{text}
Najlepiej jest zbudzić się\\
W marcu nad ranem\\
Kiedy cieniutkie czarne widełki sadu\\
Przez wiatr wygięte\\
Gdy w niedalekim lasku śnieżyczki\\
W najbielszych szatach\\
Czekają na wschód słońca by na tle czerwieni\\
Zmartwychwstać jak święte

Z dachu ubite przez czarownice\\
Biedne polskie diabły w łapciach\\
Siedząc sztywno zjeżdżają z łoskotem\\
I znikają w ziemi\\
W ogrodzie spoza zaspy uszy zajęcy\\
Widać jak wielkie otwarte nożyce\\
Na oknie głowa kota jak wielka furażerka pilnie słucha\\
Co dzieje się w sienie

Jodły w mroku majaczą\\
Ogrodu pilnują\\
A z rynny jak z długiego starego buta gdzie skąpiec\\
Chował pieniądze\\
Sypią się bez przerwy monety\\
Czerniejącą dziurą\\
Bo na odwilży lodowy buta spód spękał,\\
Życie zakończył

I rozpoznać już można góry\\
Niskie jak dach chatki nad którym wrony\\
Jak sczerniałe ze starości poszycie\\
Wirują z drzewa na wietrze\\
I widać jak już pod oknem\\
Po bieli\\
Skaczą pierwsze zielone trawki to tu to tam\\
Jak świerszcze

Najlepiej jest zbudzić się w marcu nad ranem...
\end{text}
\begin{chord}
    G C\\
    G C\\
    D e C\\
    G\\
    G C\\
    G C\\
    D e C\\
    G

    a e\\
    C G\\
    D e\\
    C G\\
    a e\\
    C G\\
    D e\\
    C G

    G C\\
    G C\\
    D e C\\
    G\\
    G C\\
    G C\\
    D e C\\
    G

    a e\\
    C G\\
    D e\\
    C G\\
    a e\\
    C G\\
    D e\\
    C G
\end{chord}