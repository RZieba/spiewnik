\tytul{Złockie Niebo Cerkiewne}{sł.Jerzy Harasymowicz, muz.Grzegorz Smiałowski}{Bez Zobowiązań}
\begin{text}
    \hfill\break
W dziupli ikon drzemią\\
Bezrobotni prorocy\\
Zasłaniają się od świata\\
Liliją Świętą\\
I planety podobne\\
Do słoneczników krążą\\
Modrą kopułą cerkiewną

\vin Jest tam i księżyc z bródką\\
\vin Chudy jak diasek\\
\vin I jest komet twarz smagła\\
\vin Z czerwonymi wstęgami\\
\vin I słońca lwia głowa\\
\vin I gwiazd modry piasek\\
\vin I płynie Święta Olga\\
\vin Z ogromnymi oczami

I krążą planety\\
I trwa Łemków niebo\\
Mocno tkwią w ziemi\\
Cerkiewne korzenie\\
Któż zatrzyma w jego jastrzębim locie\\
Słońce Łemków\\
Któż zatrzyma ich mają, huczącą\\
jak trzmiel ziemię
\end{text}
\begin{chord}
    \textit{Capo III}\\
    D C^5 G D\\
    G fis^7 h A^7\\
    D C^5 G D\\
    G fis^7 h^7\\
    e fis\\
    h^7 A^2\\
    G A D

    h A\\
    G D\\
    C^5 G\\
    D A\\
    h A\\
    G D\\
    e fis\\
    G A D
\end{chord}