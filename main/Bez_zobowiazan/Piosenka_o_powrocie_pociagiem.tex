%%
%% Author: bartek.rydz
%% 06.02.2019
%%
% Preamble
\tytul{Piosenka o powrocie pociągiem}{sł. R.Marcinkowski, muz. G.Śmiałowski}{Bez Zobowiązań}
\begin{text}
    \hfill\break
    Jeszcze się miasto kolebie w zwrotnicach\\
    za wagonem rozerwana przestrzeń;\\
    z mrowia torowisk znowu mi umykasz |bis\\
    bezosobowo i jakże pośpiesznie.

    W końcu – pociągu – wiem, że wszystko płynie;\\
    w kolejach życia znikają drobiazgi,\\
    lecz dokąd płynąć, gdy się ciągną szyny |bis\\
    i tęsknię tyłem do kierunku jazdy?

    W kurzawie znikam – z dalekiego życia\\
    patrzę w tor ślepy niewidzącym wzrokiem.\\
    Pustką wolności semafor zakwita: |bis\\
    niczyich oczu, skąd odjechał pociąg.

    Dwoista dróg powrotnych jest tęsknota:\\
    nieodwracalna, chociaż odwrócona.\\
    W dwie strony naraz płynie moje okno |bis\\
    gdy długą prostą kocham w nieskończoność.

    \hfill\break
    W studnię łoskotu i w zakręt tunelu\\
    poleciał kroplą światła ziemski padół...\\
    ...znów dzień. Jak zjawa. Porażenie bielą, |bis\\
    równina myśli i kropka w pejzażu.

    Droga spełniona. Pasaż fortepianu klas\\
    biegnie w gałęziach drzew i nic nie zmieni\\
    marzeń, co padły w głębię dokonaną |bis\\
    i utonęły w toru oddaleniu.
\end{text}
\begin{chord}
    h G x4\\
    h G\\
    D E\\
    G A h E\\
    G Fis h

    h G\\
    D E\\
    G A h E\\
    G Fis h

    h G\\
    D E\\
    G A h E\\
    G Fis E (h)

    h G\\
    D E\\
    G A h E\\
    G Fis h\\
    EGD(Fis)x4

    h G\\
    D E\\
    G A h E\\
    G Fis h

    h G\\
    D E\\
    G A h E\\
    G Fis E (h)
\end{chord}