%%
%% Author: bartek.rydz
%% 07.02.2019
%%
% Preamble
\tytul{Rybaczka}{sł. Z. Gach, muz. J. Jakubowski}{Gdańska Formacja Szantowa}
\begin{text}
    Z rodzinnych przekazów historię tę znam:\\
    O moim pradziadku, kupcu ze sfer,\\
    Co w Gdańsku wypatrzył wśród gotyckich bram\\
    Kaszubską rybaczkę z miasta Hel.\\
    Pradziadek czterdziestu dobiegał już lat\\
    (srebro mu z wolna się kładło na skroń),\\
    Rybaczka zaś - młoda i piękna jak kwiat -\\
    Miała oczy zielone jak morska toń.

    \vin Takich oczu bez dna szukał mój dziad\\
    \vin Ze świecą po świecie od zawsze;\\
    \vin Wypatrywał dniem nocą i czekał, że dla\\
    \vin Niego Amor też będzie łaskawszy.

    Pozdrowił rybaczkę, gdy zrównał z nią krok,\\
    Przedstawił się jako człowiek ze sfer;\\
    Lecz panna nieśmiało spuściła swój wzrok\\
    I umknęła na łodzi ku miastu Hel.\\
    Dzień później do Helu przepyszny wszedł jacht\\
    (złocony ornament na burtach mu lśnił):\\
    Pradziadek przypłynął odszukać ją… Ach!
\end{text}
\begin{chord}

\end{chord}