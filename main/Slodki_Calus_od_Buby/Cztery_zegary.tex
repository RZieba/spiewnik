%%
%% Author: bartek.rydz
%% 07.02.2019
%%
% Preamble
\tytul{Cztery Zegary}{sł. Z. Gach, muz. J. Jakubowski}{Gdańska Formacja Szantowa}
\begin{text}
    Na kominku z Carrary\\
    Stoją cztery zegary,\\
    Z których każdy inaczej wskazywał mój czas;\\
    Metalowe „cykadła”\\
    Przywołują widziadła\\
    Czterech dziewczyn, co sobą przesłaniały mi świat

    \vin Cztery zegary, cztery zegary\\
    \vin O wskazówkach rzeźbionych a jour;\\
    \vin Wschody, zachody, złotych godzin talary:\\
    \vin Uniesienia, westchnienia, l’amour.

    Pierwszą była Francuzka,\\
    Miała oczy jak lustra\\
    I pieprzyk maleńki…, ech, nie powiem wam, gdzie.\\
    Znała tęsknot przyczynę,\\
    Rozpieszczała mnie winem,\\
    Ale morze zazdrośnie szemrało o brzeg.

    Druga była z Jamajki,\\
    Miała kształty jak z bajki,\\
    Całowała zmysłowo, aż burzyła się krew.\\
    Założyła mi cumę,\\
    Rozpieszczała mnie rumem,\\
    Ale morze zazdrośnie szemrało o brzeg.

    Trzecia była Japonką,\\
    Miała cienkie kimonko,\\
    Gotowała wywary miłośnikom na lep.\\
    Była kwiatem i ptakiem,\\
    Rozpieszczała mnie sake,\\
    Ale morze zazdrośnie szemrało o brzeg.

    Czwarta była gdynianką,\\
    Przynosiła śniadanko\\
    Do pościeli, gdzie rzadko myśleliśmy o śnie.\\
    Dzisiaj także jest przy mnie,\\
    A gdy ręką w stół rymnie,\\
    Zapominam, że morze chlupoce o brzeg.
\end{text}
\begin{chord}
    A\\
    D\\
    A fis h E\\
    A\\
    D\\
    A h e A

    D G\\
    D h e A\\
    D G\\
    D A D G

    A\\
    D\\
    A fis h E\\
    A\\
    D\\
    A h e A

    A\\
    D\\
    A fis h E\\
    A\\
    D\\
    A h e A

    A\\
    D\\
    A fis h E\\
    A\\
    D\\
    A h e A

    A\\
    D\\
    A fis h E\\
    A\\
    D\\
    A h e A
\end{chord}