\tytul{Nad rzeką}{sł. muz. Krzysztof Jurkiewicz}{Słodki Całus od Buby}
\begin{text}
Policjanci nad rzeką\\
Odbierają siedzącym kanapki\\
Nie są pewni, dlaczego tak siedzą\\
I gdzie czai się zło\\
Tyle lat już wędruję\\
Jakoś nigdy nie mogłem tu trafić\\
Teraz wreszcie trafiłem\\
By nade mną mógł odbyć się sąd

\vin Gdybym choć raz nie był tak pijany\\
\vin Gdybym choć raz nie był pijany

Gdy frunąłem na miastem\\
Prawie wcale nie czułem obawy\\
Lewitować jest łatwo\\
Kiedy w usta całuje cię noc\\
Czy wiesz jak mam na imię\\
Cichy głos przesłuchanie prowadzi\\
Jakie ma to znaczenie\\
Jak nazywać cię mógł inny ktoś

\vin Gdy Bóg nazwał dzień dniem\\
\vin Ten na dobre wyłonił się z mroku\\
\vin Gdybym teraz mógł nadać ci imię\\
\vin Sama wiesz, żeby nie było by dla nas odwrotu

Mam na swoją obronę\\
Kilka dźwięków z pękniętej gitary\\
W swojej władzy mnie miał\\
Kołyszący się zalotnie krok\\
By zaplątać się w twoje włosy\\
Kiedyś brakło odwagi\\
Wiem, że wyrok już zapadł\\
Odczytałem go z ruchu twych rąk
\end{text}
\begin{chord}
C\\
F^{7+}\\
C\\
F^{7+}\\
C\\
F^{7+}\\
C\\
F^{7+}

G F^{7+}\\
G F^{7+}\\
(C F^{7+})\\
C\\
F^{7+}\\
C\\
F^{7+}\\
C\\
F^{7+}\\
C\\
F^{7+}

G\\
GF\\
G\\
F^{7+} G
\end{chord}