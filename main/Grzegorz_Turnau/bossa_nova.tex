%%
%% Author: bartosz.rydz
%% %28.05.2018
%%
\tytul{Bossanova sollanova}{}{Grzegorz Turnau}
\begin{text}
    A ja jestem dzieckiem smutku.\\
    Płyną łzy i pomalutku\\
    zastygają w soli grudki.\\
    Takie grudki, jakie smutki.\\
    Gdy za dużo smutków wszelkich\\
    lub gdy jeden – za to wielki,\\
    to tryskają łez fontanny\\
    i wnet stygną w kształt Solanny.

    Do-re-mi-fa-sol – Solanna,\\
    najsolenniej łez zachłanna.\\
    Sól w krąg sypie, niby manna,\\
    gdy się sączy nieustanna\\
    bossa nova solannowa\\
    solannowa bossa nova\\
    Jestem w twoim oku solą,\\
    samotnica – tańczę solo.

    Każdy dotknąć mnie się lęka,\\
    bo się rozsypuję w rękach.\\
    Czasem zaś jak Lota żona\\
    stoję w deszczu zamyślona,\\
    rozpuszczając się powoli.\\
    Ginę z braku parasoli...

    Łaknę łez, nie słodkich dżdży –\\
    słono płacę za te łzy –\\
    więc we łzach, kochany, toń,\\
    a ja chwycę twoją dłoń,\\
    wtedy weźmie nas na hol\\
    nova bossa – anna sol.
\end{text}
\begin{chord}

\end{chord}