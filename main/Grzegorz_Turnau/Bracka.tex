%%
%% Author: bartosz.rydz
%% %28.05.2018
%%
\tytul{Bracka}{}{Grzegorz Turnau}
\begin{text}
    \small{
    Na północy ściął mróz\\
    z nieba spadł wielki wóz\\
    przykrył drogi pola i lasy\\
    myśli zmarzły na lód\\
    dobre sny zmorzył głód\\
    lecz przynajmniej się można przestraszyć|

    na południu już skwar\\
    miękki puch z nieba zdarł\\
    kruchy pejzaż na piasek przepalił\\
    jak upalnie mój boże\\
    lecz przynajmniej być może\\
    wreszcie byśmy się tam zakochali

    a w Krakowie na brackiej pada deszcz\\
    gdy konieczność istnienia trudna jest do zniesienia\\
    w korytarzu i w kuchni pada też\\
    przyklejony do ściany zwijam mokre dywany\\
    nie od deszczu mokre lecz od łez

    na zachodzie już noc\\
    wciągasz głowę pod koc\\
    raz zasypiasz i sprawa jest czysta\\
    dłonie zapleć i złóż\\
    nie obudzisz się już\\
    lecz przynajmniej raz możesz się wyspać

    jeśli wrażeń cię głód\\
    zagna kiedyś na wschód\\
    nie za długo tam chyba wytrzymasz\\
    lecz na wschodzie przynajmniej życie płynie zwyczajnie\\
    słońce wschodzi i dzień się zaczyna

    a w Krakowie na brackiej pada deszcz\\
    przemęczony i senny zlew przecieka kuchenny\\
    kaloryfer jak mysz się poci też\\
    z góry na dół kałuże przepływają po sznurze\\
    nie od deszczu mokrym lecz od łez

    bo w Krakowie na brackiej pada deszcz\\
    gdy zagadka istnienia zmusza mnie do myślenia\\
    w korytarzu i w kuchni pada też\\
    przyklejony do ściany zwijam mokre dywany\\
    nie od deszczu mokre lecz od łez\\
    bo w Krakowie na brackiej pada deszcz
    }
\end{text}
\begin{chord}
    \small{
    gis Fis\\
    H E\\
    G D A2 a\\
    D G\\
    H e\\
    F E

    gis Fis\\
    H E\\
    G D A2 a\\
    D G\\
    H e\\
    F E

    a G F G\\
    F G d B\\
    a G F G\\
    F G d B\\
    a G F G
    }
\end{chord}