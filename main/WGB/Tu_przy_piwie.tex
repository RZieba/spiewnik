%%
%% Author: bartek.rydz
%% 29.05.2018
%%
% Preamble
\tytul{Tu przy piwie}{}{Wolna Grupa Bukowina}
\begin{text}
    Tu przy piwie się snują rozmowy\\
    Tak jak dym zawieszony nad stołem\\
    Leniwie słowo za słowem\\
    W gwar odpływa co naokoło\\
    Wtem poeta się zrywa do lotu\\
    Pani Stasi, co z kuflem się zbliża\\
    Barwy słów, pełni serca łopotał\\
    To jakby żyto kto ziarnem zagryzał\\
    A tam pole jak pszczele plastry\\
    Późne żniwa więc ludźmi rojne\\
    I rozmyty jesienią przestwór\\
    I powietrze na oczach szkliste

    Mieni, mieni barwami się obraz\\
    W pozłacanych ramach dzieciństwa\\
    Tu brzmi kroków echo samotne\\
    I jak ćma, co do światła się garnie\\
    Zabiega drogę przechodniom\\
    Których cienie się snują po bramach\\
    Wpław przez miasto od brzegu do brzegu\\
    Patrzy niebo przez monokl księżyca\\
    Inny księżyc się śni idącym\\
    To jakby żyto kto ziarnem zagryzał\\
    A tam pole jak pszczele plastry\\
    Późne żniwa więc ludźmi rojne\\
    I rozmyty jesienią przestwór\\
    I powietrze na oczach szkliste

    Mieni, mieni barwami się obraz\\
    W pozłacanych ramach dzieciństwa\\
    Tu nad ranem sen lepki i mokry\\
    Lęk przeszywa stukaniem od okna\\
    Wraz z głową kocem się okryć\\
    I dać głowie cichą samotność\\
    I nie myśleć o jutrze, o wczoraj\\
    Chłód od okna, jak pies stopy liże\\
    Jak zachować zamglony już obraz\\
    To jakby żyto kto ziarnem zagryzał\\
    A tam pole jak pszczele plastry\\
    Późne żniwa więc ludźmi rojne\\
    I rozmyty jesienią przestwór\\
    I powietrze na oczach szkliste

    Mieni, mieni barwami się obraz\\
    W pozłacanych ramach dzieciństwa
\end{text}
\begin{chord}

\end{chord}