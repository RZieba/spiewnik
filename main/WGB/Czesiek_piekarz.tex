%%
%% Author: bartek.rydz
%% 24.05.2018
%%

% Preamble
\tytul{Ballada o Cześku Piekarzu}{sł. i muz. Wojciech Bellon}{Wolna Grupa Bukowina}
\begin{text}
    Chleba takiego jak ten od Cześka\\
    Nie kupisz nigdzie nawet w Warszawie\\
    Bo Czesiek piekarz nie piekł lecz tworzył\\
    Bochny jak z mąki słonecznej kołacze\\
    Kłaniali mu się ludzie gdy wyjrzał\\
    Przez okno w kitlu łyknąć powietrza\\
    A kromkę masłem smarując każdy mówił -\\
    Nad chleby ten chleb od Cześka

    \vin Chleb się chlebie chleb się chlebie\\
    \vin Ponad chleb być może co\\
    \vin Chleb się chlebie chleb się chlebie\\
    \vin Nich ci nigdy nie zabraknie\\
    \vin Drożdży wody rąk i ziarna\\
    \vin (mruczał Czesiek tak noc w noc)

    A o porankach chlebem pachnących\\
    Gdy pora idzie spać na piekarzy\\
    Zaczerwienione przymykał oczy\\
    Czesiek i siadał z dłutem przy stole\\
    Ciągle te same włosy i trochę\\
    Za duży nos w drewnie cierpliwym\\
    Pieściły ręce dziesiątki razy\\
    W poranki świeżym chlebem pachnące

    Kurła tobi maty była, Kurła tobi maty była,\\
    krzyczał piekarz w żyłach krwią\\
    Kurła tobi maty była, Kurła tobi maty była\\
    Myśli moje niespokojne, myśli moje rozognione,\\
    idźcie, idżcie wszystkie stąd

    Nikt takich słów jak miasto miastem\\
    Nie znał i -"źle się dzieje" - mówili\\
    Na obraz czerniał Czesiek razowca\\
    Kruszał podobnie bułce zleżałej\\
    Gdy go znaleźli na pasku z wojska\\
    Dłuto jak wbite w bochen miał w garści\\
    I nie wie nikt co Cześka wzięło\\
    Lecz śpiewa każdy jak miasto miastem
\end{text}
\begin{chord}

\end{chord}