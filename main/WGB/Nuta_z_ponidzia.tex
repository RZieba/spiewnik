%%
%% Author: bartek.rydz
%% 29.05.2018
%%
% Preamble
\tytul{Nuta z Ponidzia}{}{Wolna Grupa Bukowina}
\begin{text}
    Polami, polami, po miedzach, po miedzach,\\
    Po błocku skisłym, w mgłę i wiatr\\
    Nie za szybko, kroki drobiąc\\
    Idzie wiosna, idzie nam. (2x)

    Rozłożyła wiosna spódnicę zieloną,\\
    Przykryła błota bury błam\\
    Pachnie ziemia ciałem młodym\\
    Póki wiosny, póki trwa. (2x)

    Rozpuściła wiosna warkocze kwieciste\\
    Zbarwiały łąki niczym kram\\
    Będzie odpust pod Wiślicą\\
    Póki wiosna, póki trwa. (2x)

    Ponidzie wiosenne, Ponidzie leniwe,\\
    Prężysz się jak do słońca kot,\\
    Rozciągnięte po tych polach,\\
    Lichych lasach w pstrych łozinach,\\
    Skałkach w słońcu rozognionym,\\
    Nidą w łąkach roziskrzoną\\
    Na Ponidziu wiosna trwa. (2x)\\
    Na Ponidziu.
\end{text}
\begin{chord}
    a F G C\\
    d G C\\
    h^7 E^7\\
    a G F^{7+} e^7\\
    a F E^7 a

    a F G C\\
    d G C\\
    h^7 E^7\\
    a G F^{7+} e^7
\end{chord}