%%
%% Author: bartek.rydz
%% 24.05.2018
%%

% Preamble
\tytul{Bukowina II}{}{Wolna Grupa Bukowina}
\begin{text}
    Dość wytoczyli bań próżnych przed domy kalecy,\\
    Żyją, jak żyli, bezwolni, głusi i ślepi.\\
    Nie współczuj, szkoda łez i żalu,\\
    Bezbarwni są, bo chcą być szarzy.\\
    Ty wyżej, wyżej bądź i dalej\\
    Niż ci, co się wyzbyli marzeń.

    Niechaj zalśni Bukowina w barwie malin,\\
    Niechaj zabrzmi Bukowina w wiatru szumie,\\
    Dzień minął, dzień minął, nadszedł wieczór,\\
    Świece gwiazd zapalił,\\
    Siadł przy ogniu, pieśń posłyszał i umilkł.

    Po dniach zgiełkliwych,\\
    po nocach wyłożonych brukiem\\
    W zastygłym szkliwie\\
    gwiazd neonowych próżno szukać\\
    Tego, co tylko zielonością\\
    Na palcach zaplecionych drzemie.\\
    Rozewrzyj dłonie mocniej, mocniej\\
    Za kark chwyć słońce, sięgnij w niebo

    Odnaleźć musisz, gdzie chmury górom dłoń podają,\\
    Gdzie deszcz i susza,\\
    gdzie lipce, październiki, maje\\
    Stają się rokiem, węzłem życia,\\
    Swój dom bukowy, zawieszony\\
    U nieba pnia, kroplą żywicy,\\
    Błękitny, złoty i zielony
\end{text}
\begin{chord}
    C d F C\\
    C d F C\\
    d G e\\
    d G C a\\
    e F Fis G C\\
    d G G^7 C

    C F G\\
    C F G\\
    C F C C^7\\
    F Fis G\\
    C d F C
\end{chord}