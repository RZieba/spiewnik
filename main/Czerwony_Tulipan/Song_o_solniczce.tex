%%
%% Author: bartek.rydz
%% 28.05.2018
%%
% Preamble
\tytul{Song o solniczce}{}{Czerwony Tulipan}
\begin{text}
    Co my tak dzisiaj, słono, gorzko\\
    głupio jest przecież głupio płakać\\
    cóż, że przedwiośnie zmienił Styczeń\\
    głos przedwczesnego umilkł ptaka\\
    I tylko o nią nasz powszedni\\
    kartofel woła, chleb i kasza\\
    bo myśmy z soli, a nie z roli\\
    ona ojczyzna nasza.\\
    Z kryształów soli, a nie z gliny\\
    Stwórca nas lepił dnia szóstego.\\
    I odszedł, sól nam pali wargi,\\
    gdy dłoń liżemy Pana swego.

    \vin Lizu, lizu wyliżemy,\\
    \vin Gryzu, gryzu wygryziemy,\\
    \vin merdu, merdu zamerdamy,\\
    \vin i tak się trzymamy.

    Po cóż z historią rozrachunki.\\
    O nowym trzeba myśleć z troską.\\
    Tamto załatwią pocałunki,\\
    A więc panowie gorzko, gorzko\\
    I wciąż głos jeden słychać wielki:\\
    ,,Kupcie cukierki, kupcie cukierki’’\\
    Więc jeszcze tylko jeden cud,\\
    i o słodycze woła lud.

    Spróbuj jeszcze raz przekrzyczeć,\\
    dławiącą w gardle, lepką słodycz.\\
    Gdy lukier usta ci okleja,\\
    a słowa grzęzną w sztucznym miodzie.\\
    Po cóż nam teraz rozrachunki.\\
    O nowym trzeba myśleć z troską.\\
    Nowe, więc słychać pocałunki.\\
    Panowie gorzko\\
    gorzko

    Lizu, lizu wyliżemy,\\
    Gryzu, gryzu wygryziemy,\\
    merdu, merdu zamerdamy,\\
    i tak się trzymamy.
\end{text}
\begin{chord}
    F\\
    D^7\\
    g C\\
    F C\\
    g C\\
    F\\
    g C\\
    F
\end{chord}