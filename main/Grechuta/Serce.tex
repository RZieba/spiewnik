%%
%% Author: bartek.rydz
%% 16.02.2019
%%
% Preamble
\tytul{Serce}{sł. Lucyna Wiśniowska, Andrzej Nowicki, muz. Jan Kanty Pawluśkiewicz}{Marek Grechuta}
\begin{text}
    Za smutek mój, a pani wdzięk\\
    ofiarowałem pani pęk czerwonych melancholii.\\
    I lekkomyślnie dałem słowo,\\
    że kwiat kwitnie księżycowo, a liście mrą srebrzyście.\\
    Pani zdziwiona mówi: 'Cóż, to przecież bukiet zwykłych róż.'\\
    Tak, rzeczywiście. Więc cóż Ci dam?

    Dam Ci srece szczerozłote.\\
    Dam konika cukrowego.\\
    Weź to serce, wyjdź na drogę.\\
    I nie pytaj się 'Dlaczego?'\\
    Weź to serce, wyjdź na drogę.\\
    I nie pytaj się 'Dlaczego?'

    Stara baba za staraganem wyrzuciła wielki kosz.\\
    Popatrz jak na złota drogę\\
    twój cukrowy konik skoczył.\\
    Popatrz jak na złotą drogę\\
    twój cukrowy konik skoczył.

    Za smutek mój, a pani wdzięk\\
    ofiarowałem pani pęk czerwonych melancholii\\
    A pani? 'Cóż, nie chce tych róż.'\\
    Że takie brzydkie, że czerwone i że z kolcami\\
    Więc cóż Ci dam?

    Dam pierścionek z koralikiem.\\
    Ten niebieski jak twe oczy.\\
    Popatrz jak na złotą drogę\\
    twój cukrowy konik skoczył.\\
    Popatrz jak na złotą drogę\\
    twój cukrowy konik skoczył.

    Dam Ci srece szczerozłote.\\
    Dam konika cukrowego\\
    Weź to serce, wyjdź na drogę.\\
    I nie pytaj się 'Dlaczego?'\\
    Weź to serce, wyjdź na drogę.\\
    I nie pytaj się 'Dlaczego?'
\end{text}
\begin{chord}

\end{chord}