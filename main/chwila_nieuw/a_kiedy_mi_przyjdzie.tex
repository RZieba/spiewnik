\tytul{A kiedy mi przyjdzie}{sł. M. Konopnicka, muz. A. Czech, Ł. Czech, M. Kuczera}{Chwila Nieuwagi}
\begin{text}
    \hfill\break
A kiedy mi przyjdzie zagrać\\
W polu dziewczynie,\\
- Da dana!\\
W polu dziewczynie,\\
Wykręcę ja fujareczkę\\
W tej wodnej trzcinie,\\
- Da dana!\\
W tej wodnej trzcinie.

Bo ta trzcina się ugina\\
Za wiatru wianiem;\\
Nie inaksza i dziewczyna\\
Z swoim kochaniem...\\
Wprawiłaby jasne słonko\\
W szybkę u chaty,\\
Wyglądałaby, czy jedzie\\
Do niej bogaty!

A kiedy mi przyjdzie zagrać\\
Na cudzym progu,\\
- Da dana!\\
Na cudzym progu,\\
Wykręcę ja fujareczkę\\
W tym ostrym głogu,\\
- Da dana!\\
W tym ostrym głogu!

Będziesz śpiewać mi przy sercu\\
Gorzkimi łzami,\\
Aż poleci głos żałosny\\
Tymi łanami...\\
Chodzi krzywda popod lasem,\\
Po jarach chodzi,\\
Nikt nie zgadnie, co za czasem\\
Złe ziarno zrodzi...

A kiedy mi przyjdzie zagrać\\
W tym pańskim dworze,\\
- Da dana!\\
W tym pańskim dworze,\\
Wykręcę ja fujareczkę\\
W najgęstszym borze,\\
- Da dana!\\
W najgęstszym borze!

Pamiętają stare drzewa\\
Tę czarną dolę,\\
Co tu przeszła przez te chaty\\
I przez to pole...\\
Rozśpiewająż się piosenki\\
W jęki i płacze,\\
Aż ściemnieją złote miody,\\
Białe kołacze...
\end{text}
\begin{chord}
    \textit{Capo II}\\
    C G C\\
    C\\
    C\\
    G C\\
    C G C\\
    C\\
    C\\
    G C

    e F C\\
    C\\
    e F C\\
    C\\
    e F C\\
    C\\
    e F C\\
    C

    C G C\\
    C\\
    C\\
    G C\\
    C G C\\
    C\\
    C\\
    G C

    e F C\\
    C\\
    e F C\\
    C\\
    e F C\\
    C\\
    e F C\\
    C

    C G C\\
    C\\
    C\\
    G C\\
    C G C\\
    C\\
    C\\
    G C

    e F C\\
    C\\
    e F C\\
    C\\
    e F C\\
    C\\
    e F C\\
    C
    \end{chord}
\begin{text}
A kiedy mi przyjdzie zagrać\\
Tej nocce śpiącej,\\
- Da dana!\\
Tej nocce śpiącej,\\
Wykręcę ja fujareczkę\\
Z wierzby płaczącej,\\
- Da dana!\\
Z wierzby płaczącej...

Oj, polecąż z niej piosenki\\
Jako ptaszkowie,\\
Oj, rozniosą ciche szumy\\
W onej dąbrowie...\\
Ni ja roli, ni ja chaty,\\
Nędzny koniucha,\\
Nocka tylko patrzy na mnie -\\
I pieśni słucha...
\end{text}
\begin{chord}
    C G C\\
    C\\
    C\\
    G C\\
    C G C\\
    C\\
    C\\
    G C

    e F C\\
    C\\
    e F C\\
    C\\
    e F C\\
    C\\
    e F C\\
    C
\end{chord}