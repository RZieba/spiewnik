\tytul{Pedziała mi matka}{}{Chwila nieuwagi}
\begin{text}
Pedziała mi matka,\\
Co by se dać pozór.\\
Na gryfnych chłopaków,\\
Broń cię, broń cię Boże,\\
Jak cię taki obłapi,\\
Żodyn niy pomoże.

W karczmie grali muzykanty,\\
Przeca są już ostatki,\\
Tańcowali wszyscy we wsi,\\
Ludzi jako szarańczy!\\
Wnet jak przyszli szykowniejsi,\\
Niźli piękny łobrozek,\\
Wszystkie wtedy postawały,\\
Wszystkie by się zabić dały.

Pedziała mi matka,\\
Co by se dać pozór\\
Na gryfnych chłopaków,\\
Broń cię, broń cię Boże,\\
Jak cię taki obłapi, \\
Żodyn niy pomoże.

Wzięli se dwie gryfne Hanki,\\
W kółko je obracali,\\
Cołki wieczór tańcowali,\\
Aż całkiem omotali!\\
Hneda obie skołowane,\\
Hanki wyszeptały:\\
My by za was duszę dały,\\
Bydymy wam wiecznie przoły.

Pedziała mi matka,\\
Co by se dać pozór\\
Na gryfnych chłopaków,\\
Broń cię, broń cię Boże,\\
Jak cię taki obłapi, \\
Żodyn niy pomoże.

Kiy yno przysięga cało,\\
Z usteczek młodych padła,\\
Naraz ze szczewika,\\
Kopyto szpetne wylazło!\\
Muzykanty grać przestali,\\
Siarką wonio srogo,\\
Dymu pełno w całej izbie,\\
A spod mantla wylozł łogon!

Pedziała mi matka,\\
Co by se dać pozór\\
Na gryfnych chłopaków,\\
Broń cię, broń cię Boże,\\
Jak cię taki obłapi, \\
Żodyn niy pomoże.

Pedziała mi matka,\\
Co by se dać pozór\\
Na gryfnych chłopaków,\\
Broń cię, broń cię Boże,\\
Jak cię taki obłapi, \\
Diobeł to być może.
\end{text}
\begin{chord}
\end{chord}