%%
%% Author: bartosz.rydz
%% %28.05.2018
%%
\tytul{Dzwony}{sł M. Konopnicka, muz. Martyna i Andrzej Czech}{Chwila Nieuwagi}
\begin{text}
    A gdy skonał w czarnej chacie\\
    Jasieńko miły,\\
    Poszła matka prosić dzwonów,\\
    By mu dzwoniły.

    — Mój synaczek, mój rodzony,\\
    W trumience leży,\\
    O zagrajcież wy mu, dzwony,\\
    Z tej białej wieży!

    Niechaj idzie głos bijący\\
    O jasne słońce,\\
    Przez te pola, przez te lasy,\\
    Z wiatrem szumiące...

    Ale dzwony twarde serca,\\
    Zimną pierś miały.\\
    Będziem jemu dzwonić, matko,\\
    Za talar biały. —

    I wróciła, narzekając,\\
    Do pustej chaty\\
    I strząsnęła wszystkie kąty\\
    I zgrzebne szmaty...

    I nic więcej nie znalazła\\
    Prócz onej świty,\\
    Którą syna trup sczerniały\\
    Leżał nakryty...

    — Nieszczęśliważ moja dola,\\
    Jasieńku miły!\\
    Chybaż tobie łzy te moje\\
    Będą dzwoniły...

    Chyba moje narzekanie\\
    Bić będzie z rosą,\\
    Kiedy ciebie na mogiłki\\
    Z chaty wyniosą!

    I wynieśli za próg czarny\\
    Trumienkę lichą,\\
    A za synem poszła matka\\
    Ścieżyną cichą...

    I nie grały jemu dzwony\\
    Z wysokiej wieży,\\
    Jeno szumiał las zielony\\
    I wietrzyk świeży...
\end{text}
\begin{chord}

\end{chord}