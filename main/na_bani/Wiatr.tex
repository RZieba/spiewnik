\tytul{Wiatr dla dzieci}{Kazimierz Wierzyński}{Dnieje}
\begin{text}
Pokłóciły się drzewa, wiatr im rozum pomieszał,\\
Może liście bukowe u brzóz pozawieszał.\\
Może sosny pomylił z topolami i łozą,\\
Teraz gonne, masztowe do tartaku powiozą.\\
Może wszystko połamał, z korzeniami wykopał,\\
Cały las wydziedziczył i przeznaczył na opał.

Bo wyrywa wciąż włosy, rozum traci i wieje\\
I naprawdę nikt nie wie, co z tym wiatrem się dzieje.\\
Tylko w górze nad lasem, coraz czarniej i czarniej.\\
Pewnie smołę tam rozlał i podpalił smolarnię.\\
I mgły się zapaliły i żarnowce i cząbry\\
I wiatr taki wielki i taki niemądry.
\end{text}