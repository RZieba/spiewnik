\tytul{Mury Toledo}{sł. T. Borkowski, muz. Mysza Michalska}{Na Bani}
\begin{text}
    \hfill\break
Było ich dwóch, oni dwaj przetrwali\\
Jej dumne gesty, jej próżne kaprysy\\
Jej spojrzenia z toledańskiej stali\\
Zostali dwaj w cieniu cyprysów

Lecz gdy ich klingi rozbłysnęły krzyżem\\
Gdy przeciw sobie rzucili z pogardą\\
Wszystko, co każdy z nich jej kiedyś przyrzekł\\
Jeden z nich zamarł z opuszczoną gardą

Nagle zachwiały się mury Toledo\\
Tak wokół niej zwinął się w spazmach\\
Gdy zrozumiała, że chce tylko jego\\
Przeklęła imię swe Esperanza

Kiedy nadzieja wsiąkła w bruk miasta\\
Które też miało serce z kamienia\\
Jak dzwon gdy pęknie, w jednej chwili zgasła\\
Stanęła tak niemo, że niemal jej nie ma

Zerwały się struny napięte do granic\\
Nikt już przez rzekę mostów nie przerzuci\\
A przecież zaraz pobiegła by za nim\\
Nim szpada spadła i dosięgła bruku

Jak potrzask wąskie stały się ulice\\
I tylko na dół schody ją wiodą\\
W tych murach nawet nie da się krzyczeć\\
Z żadnym nie będzie i nie będzie sobą

Nagle zachwiały się mury Toledo\\
Tak wokół niej zwinął się w spazmach\\
Gdy zrozumiała, że chce tylko jego\\
Przeklęła imię swe Esperanza

Kiedy nadzieja wsiąkła w bruk miasta\\
Które też miało serce z kamienia\\
Jak dzwon gdy pęknie, w jednej chwili zgasła\\
Stanęła tak niemo, że niemal jej nie ma
\end{text}
\begin{chord}
    \textit{Capo II}\\
    E F*\\
    E F* G* F*\\
    E F*\\
    E F* E

    E F*\\
    E F* G* F*\\
    E F*\\
    E F* E

    E F*\\
    E F*\\
    E F*\\
    E G* F* E

    a G\\
    F d E\\
    a G\\
    F d E

    E F*\\
    E F* G* F*\\
    E F*\\
    E F* E

    E F*\\
    E F* G* F*\\
    E F*\\
    E F* E

    E F*\\
    E F*\\
    E F*\\
    E G* F* E

    a G\\
    F d E\\
    a G\\
    F d E
\end{chord}