\tytul{Cerkiew w ogniu}{}{Na Bani}
\begin{textn}
    Choć przetrwała gorący czas,\\
    Cerkiew w ogniu stanęła jesienią.\\
    Nie boskiej chwały to blask.\\
    Nie wiara w niej wstała płomieniem. x2

    \ifchorded{\hfill\break}
    \vin Jak sam Bóg w mroku lśni ikonostas,\\
    \vin Z nocy nagle rozbłysły ikony.\\
    \vin Złotem spływa z ich oczu rozpacz,\\
    \vin Nim się żywot świętych dokona.

    \vin Bizantyjskie czernieją twarze,\\
    \vin Otwierają oczy szeroko.\\
    \vin Święty Michał przegrywa z Szatanem,\\
    \vin Święty Jerzy pada przed smokiem. x2

    Choć przetrwała gorący czas,\\
    Cerkiew w ogniu stanęła jesienią.\\
    Nie boskiej chwały to blask.\\
    Nie wiara w niej wstała płomieniem.

    \vin Bóg wszechmocny patrzy bezsilnie,\\
    \vin Jakby tylko z obrazu był Bogiem.\\
    \vin Nie z kadzideł dym kryje mandylion,\\
    \vin To gniewu trawi go ogień.

    \vin Iskry z oczu aniołów lecą,\\
    \vin Załamują prorocy dłonie,\\
    \vin Na wszystkie świętości złorzecząc:\\
    \vin A niech was piekło pochłonie!

    Choć przetrwała gorący czas,\\
    Cerkiew w ogniu stanęła jesienią.\\
    Nie boskiej chwały to blask.\\
    Nie wiara w niej wstała płomieniem.

    \ifchorded{\hfill\break}
    W proch obrócą się czarne zgliszcza,\\
    Lecz co rok drzewa wokół goreją.\\
    Pożar przecież pamięci nie zniszczy\\
    O tej cerkwi, co zgasła jesienią.
\end{textn}
\begin{chordw}
    gis Fis gis\\
    H E Fis\\
    E Fis gis\\
    E Fis gis\\
    E Fis gis A H cis^9

    cis A\\
    fis A\\
    cis A\\
    fis A

    cis E fis gis\\
    A H\\
    cis A\\
    fis A\\
    (Cis H fis Gis)\\
    cis H cis\\
    E A H\\
    A H cis\\
    A H cis\\
    (A H cis A H)\\
    cis A\\
    fis A\\
    cis A\\
    fis A

    cis E fis gis\\
    A H\\
    cis A\\
    fis A\\
    (Cis H fis Gis)\\
    cis H cis\\
    E A H\\
    A H cis\\
    A H cis\\
    (A H cis A H)

    gis Fis gis\\
    H E Fis\\
    E Fis gis\\
    E Fis gis\\
\end{chordw}