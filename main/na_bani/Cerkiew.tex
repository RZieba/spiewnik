\tytul{Cerkiew w ogniu}{}{Na Bani}
\begin{text}
    Choć przetrwała gorący czas,\\
    Cerkiew w ogniu stanęła jesienią.\\
    Nie boskiej chwały to blask.\\
    Nie wiara w niej wstała płomieniem. x2

    Jak sam Bóg w mroku lśni ikonostas,\\
    Z nocy nagle rozbłysły ikony.\\
    Złotem spływa z ich oczu rozpacz,\\
    Nim się żywot świętych dokona.

    Bizantyjskie czernieją twarze,\\
    Otwierają oczy szeroko.\\
    \vin Święty Michał przegrywa z Szatanem,\\
    \vin Święty Jerzy pada przed smokiem. x2

    \vin Choć przetrwała gorący czas,\\
    \vin Cerkiew w ogniu stanęła jesienią.\\
    \vin  Nie boskiej chwały to blask.\\
    \vin Nie wiara w niej wstała płomieniem.

    Bóg wszechmocny patrzy bezsilnie,\\
    Jakby tylko z obrazu był Bogiem.\\
    Nie z kadzideł dym kryje mandylion,\\
    To gniewu trawi go ogień.

    Iskry z oczu aniołów lecą,\\
    Załamują prorocy dłonie,\\
    Na wszystkie świętości złorzecząc:\\
    A niech was piekło pochłonie!

    W proch obrócą się czarne zgliszcza,\\
    Lecz co rok drzewa wokół goreją.\\
    Pożar przecież pamięci nie zniszczy\\
    O tej cerkwi, co zgasła jesienią.
\end{text}
\begin{chord}
    gis Fis gis\\
    H E Fis\\
    E Fis gis\\
    E Fis gis\\
    (A H cis^9)\\
    cis E fis gis\\
    A H\\
    cis A\\
    fis A\\
    (Cis H fis Gis)\\
    Cis H Cis\\
    E A H\\
    A H cis\\
    A H cis (AH)
\end{chord}