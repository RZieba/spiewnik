\section{Rekwije dla Mistrza Jerzego}
\begin{text}
\footnotesize{Z gór wracałem, gdy Ty w góry niebieskie odszedłeś\\
Znów minąłem się z Tobą i w końcu na zawsze\\
Drogi Mistrzu, Twe drogi tyle razy przebiegłem\\
Nigdy kroków z Twoimi nie skrzyżowawszy

W pełnym barze pod stół pijane lecą już wiersze\\
Odmawiają tam goście za Ciebie piwne litanie\\
Bo poetą przedmieścia Ty byłeś i będziesz wciąż pierwszym\\
Chociaż los zdmuchnął Twoją wędrowca latarnię

Świat spod ciemnej gwiazdy w słowach zwyciężyłeś jasnych\\
Wyblakłym landszaftom z kolorów nałożyłeś mitrę\\
Leniwym chmurom w końcu bieg nadałeś własny\\
Nieujarzmionej myśli popędziłeś wichrem

Spójrz na mnie czasem z nieba okiem jastrzębim\\
Gdy zacznę znów klecić rymy jakieś kulawe\\
Lecz pozwól mi zrywać słowa z Twoich jarzębin\\
Gdy będę przemierzał Twoje wciąż i na wieki pejzaże

Z gór wracałem, gdy Ty odszedłeś w góry niebieskie\\
I zabrał Cię wóz z widokiem na Bieszczady\\
Świętym Jerzym pewnie zapamięta Cię Beskid\\
Ikony żegnać Cię będą łez Popradem

Ciągle stoi nad światem gór ikonostas czarny\\
Gdyś po drugiej już stronie, Twa dusza niech go rozjaśni\\
Tak jak wierszem kopuły starej cerkwi podparłeś\\
Spraw, by wiatru chorały nigdy nie gasły

Świat spod ciemnej gwiazdy w słowach zwyciężyłeś jasnych\\
Wyblakłym landszaftom z kolorów nałożyłeś mitrę\\
Leniwym chmurom w końcu bieg nadałeś własny\\
Nieujarzmionej myśli popędziłeś wichrem

Spójrz na mnie czasem z nieba okiem jastrzębim\\
Gdy zacznę znów klecić rymy jakieś kulawe\\
Lecz pozwól mi zrywać słowa z Twoich jarzębin\\
Gdy będę przemierzał Twoje wciąż i na wieki pejzaże}
\end{text}