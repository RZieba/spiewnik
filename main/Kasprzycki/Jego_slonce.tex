%%
%% Author: bartek.rydz
%% 28.05.2018
%%
% Preamble
\tytul{Jego słońce}{}{Robert Kasprzycki}
\begin{text}
    Siedzi w barze tym od świtu,\\
    nie stać go na pierwsze piwo,\\
    czasem ktoś go poczęstuje papierosem.\\
    Stałe miejsce tuż przy ladzie,\\
    stamtąd lepiej wszystko widać,\\
    tutaj wschodzi i zachodzi jego słońce.

    Kiedy wraca ulicami\\
    ceny mizdrzą się z wystawy,\\
    nie dla niego smak Edenu luksusowy.\\
    On kupuje odzież tanią,\\
    która sama się zachwala\\
    że już ktoś ją przenicował łzą kaprawą.

    Zwolnione godziny, rozmowy o niczym\\
    a usta wciąż mielą przekleństwa.\\
    Z tej mąki polityk wypiecze chleb gorzki,\\
    brunatny sen triumfu i klęski.

    Siedzi w barze tym do świtu,\\
    pięści puchną pod stolikiem,\\
    gdyby jeszcze raz tak złapać gdzieś robotę.\\
    Bo podobno dom budują,\\
    który lśni tysiącem okien\\
    lecz ten dom za symbolicznym stoi płotem.

    Zwolnione godziny, rozmowy o niczym,\\
    a usta wciąż mielą przekleństwa.\\
    Z tej mąki polityk wypiecze chleb gorzki,\\
    brunatny sen triumfu i klęski.

    Siedzi w barze tym od świtu,\\
    nie stać go na pierwsze piwo,\\
    czasem ktoś go poczęstuje papierosem.\\
    Stałe miejsce tuż przy ladzie,\\
    stamtąd lepiej wszystko widać,\\
    tutaj rodzi się i tu umiera\\
    jego słońce.
\end{text}
\begin{chord}

\end{chord}