%%
%% Author: bartek.rydz
%% 28.05.2018
%%
% Preamble
\tytul{Mówią o nim}{}{Robert Kasprzycki}
\begin{text}
    Tak mówią o nim jakby już nie żył\\
    choć wciąć na miękkich nogach pomyka\\
    prochów nie łyka ani nie tnie żył\\
    najwyżej wolno sączy żywczyka

    mówią już o nim jakby był skonał\\
    choć jeszcze dycha za dychą spada\\
    na ladę baru ladę nie lada\\
    ladę pod którą spadło już paru

    mówią już o nim jakby w kalendarz\\
    kopnął ze złości albo niedbalstwa\\
    lecz wciąż się włóczy jak ludzka menda\\
    po szarym świecie w mgiełce pijaństwa

    tak mówią o nim jakby wyciągnął\\
    wnioski z przesłanek oraz kopyta\\
    a przecież ślęczy w knajpie pod stągwią\\
    zbożowej wódki i dym wciąż wdycha

    tak mówią o nim jakby mu biała\\
    panna koścista uszyła buty\\
    a przecież nadal rumieńcem pała\\
    chociaż spuchnięty jest i zapluty

    tak mówią o nim jakby wyzionął\\
    ducha z kolejki której nie dopił\\
    a przecież wraca na knajpy łono\\
    każdego ranka żale swe topi

    tak mówią o nim jakby nie istniał\\
    jakby ktoś grubą skreślił go kreską\\
    choć ciągle przed nim wódeczka czysta\\
    wabi teksturą lekko niebieską

    tak mówią o nim jakby już nie żył\\
    jakby ktoś w trumnę rzucił mu proso\\
    choć on jedyny z wiernych żołnierzy\\
    na śniegu stoi nago i boso

    tak mówią o nim znów to słyszałem\\
    umarł nam Kefas leży na desce\\
    lecz on nie upadł choć spada dycha\\
    żyje i pije i dycha jeszcze
\end{text}
\begin{chord}

\end{chord}