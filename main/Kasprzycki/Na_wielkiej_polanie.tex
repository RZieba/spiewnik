%%
%% Author: bartek.rydz
%% 28.05.2018
%%
% Preamble
\tytul{Na Wielkiej Polanie}{}{Robert Kasprzycki}
\begin{text}
    Na Wielkiej Polanie, pod świętym obrazem,\\
    siedzimy dziś razem, już pachnie śniadanie,\\
    herbata, chleb, kawa, masełka osełka,\\
    Zawoja za lasem, nad Zawoją mgiełka.\\
    Deszcz płacze na niby stukając do szyby\\
    na strychu schną grzyby, na obcych pies szczeka.\\
    Prąd uciekł złośliwie z kontaktu i jak tu\\
    kefirem napoić głodnego człowieka?

    Na Wielkiej Polanie deszcz szczypie schronisko,\\
    do nieba tak blisko Panowie i Panie,\\
    uciekli od planów, komórek i miasta,\\
    za lasem wyrasta miasteczko Jordanów.\\
    Wrócili przed chwilą tak zwani wędrowcy,\\
    'swetr' prosto od owcy zawieszą nad piecem,\\
    w kontakcie napięcie szczególne ma wzięcie,\\
    na razie naprędce stosuje się świece.

    Na Wielkiej Polanie króciutka pokuta,\\
    prąd krąży już w drutach i tak już zostanie,\\
    niech dzwonią dżdżu krople i dzwony w Sidzinie,\\
    niech czas cicho płynie i zimny jak sople,\\
    niech sączy się strumień ten koło schroniska,\\
    niech sączy się cicho z daleka i z bliska,\\
    bezsenna piosenka, półświęta, półświecka,\\
    piosenka tak miękka jak policzek dziecka.
\end{text}
\begin{chord}

\end{chord}