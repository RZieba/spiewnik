%%
%% Author: bartek.rydz
%% 28.05.2018
%%
% Preamble
\tytul{Brudny autobus do stacji Golgota}{}{Robert Kasprzycki}
\begin{text}
    Gdy autobus rusza wyrzucam papierosa\\
    choć kierowcy linii całkiem podmiejskiej\\
    jakoś obojętne dokąd chcemy jechać,\\
    po co, skąd, kiedy, czemu i za jaką cenę.\\
    A bilet kosztował trzydzieści srebrników\\
    do stacji z napisem pożółkłym Golgota.\\
    Miałem sen: ktoś mój portret zawiesił na drzewie,\\
    padał deszcz, potem pękła zasłona ze złota.

    Brudny autobus do stacji Golgota,\\
    ukradkiem zerkam przez judasz przetarty\\
    palcem w szybie. Mijamy stacje i zakręty.\\
    Tak się dłuży w tej podróży jakbyśmy do nieba.

    Rżną w karty żołnierze. Piją. Będzie bieda.\\
    Chytrus Chrystus się przysiadł - fałsz winem przepija.\\
    Znów handel mu się udał, stare ciuchy sprzedał.\\
    Więc się cieszy, choć od rana coś kłuje go w krzyżu.\\
    Zapomniany autobus do stacji Golgota\\
    raptem staje - z kół uciekło ze świstem powietrze.\\
    Kiedy Chrystus pijany klnie na swego ojca\\
    w twarz od niego dostaje - w dół spada po stopniach.

    Brudny autobus do stacji Golgota\\
    ukradkiem zerkam przez judasz przetarty\\
    palcem w szybie. Mijamy stacje i zakręty.\\
    Tak się dłuży w tej podróży jakbyśmy do nieba.

    Pewnie jutro ktoś opisze incydent ten zwykły,\\
    ubarwiając apokryfem w brukowej gazecie,\\
    jakiś żul, jakiś pismak, nadworny poeta,\\
    by nie nudził się czytelnik tłusty przy obiedzie.\\
    Jutro Chrystus się doigra, zawiśnie na krzyżu,\\
    może za to, że Piłata w pokera oszwabi.\\
    Nie pojedzie nikt na Golgotę, na pogrzeb\\
    wszyscy pójdą. Potem stypa. Będzie trochę zabawy.

    Brudny autobus do stacji Golgota\\
    ukradkiem zerkam przez judasz przetarty\\
    palcem w szybie. Mijamy stacje i zakręty.\\
    Tak się dłuży w tej podróży jakbyśmy do nieba.
\end{text}
\begin{chord}

\end{chord}