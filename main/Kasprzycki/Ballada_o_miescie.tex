%%
%% Author: bartek.rydz
%% 28.05.2018
%%
% Preamble
\tytul{Ballada o mieście}{}{Robert Kasprzycki}
\begin{text}
    Dzień znów kończy się patrzeniem w sufit\\
    jego biel rozjaśnia nocny spleen\\
    miasto śpi swym snem kamiennych żółwi\\
    palcem w ciemność kreślę rejestr swoich win\\
    Znów patrzyłem wokół lecz nie zobaczyłem\\
    co być może dostrzegłby najgłupszy z was\\
    że zdeptałem to co sam stworzyłem\\
    długo budowany hermetyczny świat

    Przez brudne czyny plugawe rozmowy\\
    dzień conocny i codzienną noc\\
    zawsze winny i ciągle bez winy\\
    między prawdą a kłamstwem przenikam jak kot

    Z obrzydzeniem patrzę w otępiałe twarze\\
    bagnem bzdur bełkocze ten pijany tłum\\
    pomiędzy rynsztokiem a świętym ołtarzem\\
    płyną brudną lawą niby Styksu nurt\\
    I nierozgrzeszony w swojej nienawiści\\
    usta mam skrzywione w paroksyzmie słów\\
    że ten senny koszmar nigdy się nie wyśni\\
    bo jest przebudzeniem z koszmarnego snu

    Przez brudne czyny plugawe rozmowy\\
    dzień conocny i codzienną noc\\
    zawsze winny i ciągle bez winy\\
    między prawdą a kłamstwem przenikam jak kot
\end{text}
\begin{chord}
    a G a F\\
    a G a F\\
    a G a F\\
    a G a F\\
    a G a F\\
    a G a F\\
    a G a F\\
    a G a F

    C G F G\\
    C G F G\\
    C G F G\\
    C G FEa
\end{chord}