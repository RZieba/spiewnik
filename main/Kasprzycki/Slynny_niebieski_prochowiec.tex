%%
%% Author: bartek.rydz
%% 29.05.2018
%%
% Preamble
\tytul{Słynny niebieski prochowiec}{sł.Leonard Cohen, tłum. Maciej Zembaty}{Robert Kasprzycki}
\begin{text}
    Jest czwarta nad ranem, już kończy się grudzień.\\
    List piszę do Ciebie: Czy dobrze się czujesz?\\
    W New Yorku jest zimno, poza tym w porządku-\\
    Muzyka na Clinton Street gra na okrągło.

    Podobno budujesz swój własny dom\\
    W głębi pustyni.\\
    Od życia nie chcesz już nic,\\
    Lecz musiałeś zachować wspomnienia.

    A Jane do dziś kosmyk włosów ma Twych\\
    Wiem, że gdy dawałeś go jej\\
    Myślałeś już o tym by zwiać,\\
    Lecz niełatwo jest zwiać...

    Gdy tu byłeś ostatnio, wyglądałeś jak starzec:\\
    Podniszczyłeś swój słynny niebieski prochowiec.\\
    Do każdego pociągu wychodziłeś na dworzec-\\
    Bez swej Lili Marlen pojechałeś do domu.

    Dałeś mojej kobiecie\\
    Swego życia ledwie strzęp;\\
    Nie jest już moją żoną\\
    I Twoją też nie...

    Ciągle widzę Cię z tą różą w zębach, choć wiem,\\
    Ze to tani był greps,\\
    Lecz spodobał się Jane...\\
    Jane pozdrawia Cię też.

    Cóż mam Ci powiedzieć, mój bracie, mój kacie?\\
    Sam nie wiem, czy pisać, czy nie?\\
    Brakuje mi Ciebie, przebaczam od siebie -\\
    To dobrze żeś w drogę mi wszedł...

    A może byś tak tutaj wpadł\\
    Do mnie lub do Jane\\
    Pomyśl, Twój wróg sypia twardo\\
    A żona nudzi się

    Więc dziękuję Ci, że\\
    Wypędziłeś jej z oczu ten żal\\
    Ja myślałem, że musi być tak\\
    Nie starałem się więc

    A Jane do dziś kosmyk włosów ma Twych\\
    Wiem, że gdy dawałeś go jej\\
    Myślałeś już o tym by zwiać

    Z poważaniem\\
    Leonard Cohen
\end{text}
\begin{chord}

\end{chord}