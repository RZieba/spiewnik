%%
%% Author: bartek.rydz
%% 28.05.2018
%%
% Preamble
\tytul{Kolęda nowa}{}{Robert Kasprzycki}
\begin{text}
    I znowu nie mam piosenki na Święta,\\
    a przecież wszyscy piszą o nich łatwo,\\
    jest w tych piosenkach Mikołaj wraz z dziatwą,\\
    stół i choinka a w wannie karpięta.

    A ja nie umiem pisać o tych Świętach,\\
    a mnie od razu znosi na manowce.\\
    U innych szopka, gwiazdka, krówki, owce,\\
    opłatek, w oknie buzia uśmiechnięta.

    Kolęda, kolęda, luli, luli la,\\
    może nam Rok Nowy trochę szczęścia da.

    Przecież tak łatwo pisać o tych Świętach;\\
    o tym talerzu dla pielgrzyma w drodze,\\
    ździebełko słomy błyszczy na podłodze,\\
    proszę jak łatwo napisać o Świętach.

    A ja nie umiem pisać o tych Świętach,\\
    że wszyscy razem i przy jednym stole\\
    nad stołem Anioł, a w wiejskiej stodole\\
    już ludzkim głosem gadają zwierzęta.

    Kolęda, kolęda, luli, luli la,\\
    może nam rok nowy trochę szczęścia da.

    Bo moje Święta od lat nie są święte,\\
    Wiem, takich domów wiele na tym świecie,\\
    wciąż kogoś braknie, chciałoby się lecieć,\\
    ździebełkiem słomy wiązać dom pęknięty.

    Chcę pisać szczerze, więc nie piszę wcale,\\
    lecz dla tych wszystkich, którym w oczy wieje,\\
    kreślę życzenia, by mieli nadzieję,\\
    gdy łza jak gwiazda spadnie im na talerz.

    Kolęda, kolęda, luli, luli la,\\
    może nam rok nowy trochę szczęścia da
\end{text}
\begin{chord}

\end{chord}