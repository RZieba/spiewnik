\tytul{Życie jest dziwne}
{Artur Andrus}
{Artur Andrus}
\begin{text}
Podczas pobytu w hufcu Grodno,\\
Zastęp harcerzy z hufca Lesko,\\
Wytropił żółtą łódź podwodną,\\
I przemalował ją na niebiesko.

Wyrosły na niej łopian i mech,\\
Życie jest dziwne, eh.\\
Wyrosły na niej łopian i mech,\\
Życie jest dziwne, eh.

W środku znaleźli stół stalowy,\\
I siedem pryczy niewygodnych,\\
To taki zestaw standardowy,\\
Niektórych typów łodzi podwodnych.

I szafę zbitą z dębowych dech,\\
Życie jest dziwne, eh.\\
I szafę zbitą z dębowych dech,\\
Życie jest dziwne, eh.

Nagle w tej łodzi coś załkało,\\
Więc poruszono takim wieczkiem,\\
A tam na mostku dziecko stało,\\
W dłoni trzymało małą książeczkę.

A w drugiej portret premiera Czech,\\
Życie jest dziwne, eh.\\
A w drugiej portret premiera Czech.\\
Życie jest dziwne, \\
Życie jest dziwne, dziwne, eh.

Chłopiec na imię miał Oktawian,\\
Ojciec Jaromir, matka Zdenka,\\
Ktoś się być może zastanawia,\\
O czym naprawdę jest ta piosenka.

Kiedy Wam powiem zaprze Wam dech,\\
Życie jest dziwne, eh.\\
Kiedy Wam powiem zaprze Wam dech,\\
Życie jest dziwne, \\
Życie jest dziwne, dziwne, eh.

Idą na górę Świętej Anny,\\
Tłumy niezmierne i pobożne,\\
A to jest pieśń o nieustannym,\\
Kryzysie polskiej piłki nożnej,\\
Który tak samo mnie obchodzi,\\
Jak czeskie dziecko w podwodnej łodzi.

Bo czy to Legia, Wisła czy Lech,\\
Życie jest dziwne, eh.\\
Bo czy to Legia, Wisła czy Lech,\\
Życie jest dziwne,\\
Życie jest dziwne, dziwne.

Bardzo dziwne,\\
Bardzo, bardzo, bardzo dziwne,\\
Bardzo dziwne,\\
Bardzo, bardzo, bardzo dziwne,\\
Bardzo dziwne\\
Bardzo dziwne, jest. [x4]
\end{text}
\begin{chord}
\end{chord}