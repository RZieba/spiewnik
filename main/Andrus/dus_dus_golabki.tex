\tytul{Duś, duś gołąbki}
{Artur Andrus}
{Artur Andrus}
\begin{text}
Na Podkarpaciu, w prastarych lasach,\\
Gdzie niebo czysto niebieskie,\\
Pani Teresa była na wczasach,\\
Przez dwa tygodnie i z pieskiem.\\
A niedaleko w ceglanym bloku,\\
Mieszka ozdoba tej dziczy,\\
Brunecik z wąsem o sarnim wzroku,\\
Czyli miejscowy leśniczy.\\

Duś, duś gołąbki,\\
Jadą, jadą w gości,\\
Duś, duś gołąbki,\\
Oda do radości.\\

On by jej z drzewa nastrącał szyszek,\\
I kochał by ją najczulej,\\
Ale Teresa i leśnik Zbyszek\\
Się nie spotkali w ogóle.\\
Pani Teresa całymi dniami\\
Się wyleguje pod słońcem,\\
A on w tym lesie chemikaliami,\\
Dokarmia dzikie zaskrońce.\\

Duś, duś gołąbki,\\
Jadą, jadą w gości,\\
Duś, duś gołąbki,\\
Oda do radości.\\

Ona się lubi przyglądać żabom,\\
I czytać historię Gniezna,\\
Trochę się męczy, bo widzi słabo,\\
I wszystkich liter też nie zna.\\
On, kiedy las zaniesie się ciszą,\\
Leci jak wiatr przez rozłogi\\
Do pewnej Jadźki, o której piszą\\
Doktorat z dermatologii.\\

Smutno się skończy moja piosenka:\\
Teresa umrze od słońca,\\
Leśnik od Jadźki, w okrutnych mękach,\\
A piesek od zaskrońca,\\
Zostanie po Teresie saszetka,\\
Po psie obróżka z kolczykiem,\\
A po leśniczym mała plakietka,\\
Tak, jestem Europejczykiem.\\

Duś, duś gołąbki,\\
Jadą, jadą w gości,\\
Duś, duś gołąbki,\\
Oda do radości.\\

Odpowiedzi na zagadkę konkursowa,\\
Jak ma na imię Teresa?\\
Kierować do Nadleśnictwa Strzałowo,\\
Można też pekaesem wysłać esemesa. \\

Duś, duś gołąbki,\\
Pali się pod gruszą,\\
Duś albo nie duś,\\
Same się poduszą.\\
\end{text}