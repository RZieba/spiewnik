\tytul{Ballada o wyważonym koniu księcia Józefa}
{Artur Andrus}
{Artur Andrus}
\begin{text}
Zanim Cię wywiodą w pole\\
Krzyż Ci dadzą albo broń.\\
Popatrz stoi na cokole\\
Wyważony koń

On się na to patrzeć wstydzi,\\
Bo spod wyważonej grzywy \\
Koń to wszystko wokół widzi\\
Z innej perspektywy.\\

Koniu nisko zwisa ogon.\\
Koń się trochę zna na życie.\\
Temu koniu wszyscy mogą\\
Cmoknąć przy kopycie.\\
x2

Jakiś bunt się w koniu budzi,\\
Z niepokojem patrzy w dal.\\
Tylko ludzi, tylko ludzi,\\
Tylko ludzi żal.

Zanim ruszysz do ataku,\\
Zanim pierwszy stos zapłonie,\\
Może byś se tak rodaku\\
Porozmawiał z koniem.

Koniu nisko zwisa ogon.\\
Koń ma jakąś taką głębię.\\
Temu koniu wszyscy mogą\\
Poszczotkować w kłębie.\\
x2

Chcesz to ganiaj się po Błoniach.\\
Bij się, jeśli chcesz się bić.\\
Ale odczep się od konia.\\
Dajże koniu żyć.

W tym tkwi całej sprawy sedno.\\
Też mu ciążą nasze losy.\\
Koniu nie jest wszystko jedno,\\
Tylko ma już dosyć.

Koniu nisko zwisa ogon.\\
Inne sprawy ma na głowie.\\
Temu koniu wszyscy mogą\\
Skoczyć na pokrowiec.\\
x2

Temu koniu wszyscy mogą\\
(*cmok*)
\end{text}
\begin{chord}
D G
D G
h f# e G D

G H
e A G
D C F D
E A D

D e F F# G
\end{chord}
