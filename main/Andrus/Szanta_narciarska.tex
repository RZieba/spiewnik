%%
%% Author: bartek.rydz
%% 10.02.2019
%%
% Preamble
\tytul{Szanta narciarska}{Artur Andrus}{Artur Andrus}
\begin{text}
    \ifOneCol \else \footnotesize{ \fi
    \ifchorded{\hfill\break}
    Nazywali go marynarz, Bo opaskę miał na oku.\\
    Na każdym stoku dziewczyna,  Dziewczyna na każdym stoku.\\
    Pochodzi spod Poznania, Podobno umie wróżyć z kart.\\
    Panny rwie na wiązania, Mężatki - na długość nart.

    Caryco mokrego śniegu Ratrakiem płynę do Ciebie pod prąd.\\
    Dobrze, że stoisz na brzegu, Bo ja właśnie schodzę na ląd.

    Nigdy się nie lękał biedy I się nie przejmował jutrem.\\
    A jego ratrak był kiedyś Zwyczajnym rybackim kutrem.\\
    I woził dorsze i śledzie, Zimą i latem, okrągły rok.\\
    Teraz jak nieraz przejedzie Rybami czuć cały stok.

    Wszyscy w porcie odetchnęli Zwiał nim się zakończył sezon.\\
    Jeszcze się tam jak żagiel bieli Jego czarny kombinezon.\\
    Odpłynął pod Ustrzyki I przez kobiety wpadł w kłopoty.\\
    Forsę z polowań na orczyki Przehulał na antybiotyk.

    Jeśli kiedyś go zobaczysz Na ratraku w podłym świecie,\\
    To powiedz mu, że w Karpaczu Czekają na niego dzieci.\\
    I kiedy opuszcza statek, Żeby się znowu oddać złu,\\
    Każda z dwudziestu siedmiu matek Dzieciątku śpiewa do snu:
    \ifOneCol \else } \fi
\end{text}
\begin{chord}
    \ifOneCol \else \footnotesize{ \fi
    \textit{Cap II}\\
    d a d C F\\
    g d B a d\\
    d a d C F\\
    g d B a d

    d a d g\\
    g d B a d
    \ifOneCol \else } \fi
\end{chord}