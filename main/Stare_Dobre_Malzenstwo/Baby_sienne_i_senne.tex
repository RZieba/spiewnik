%%
%% Author: bartek.rydz
%% 16.02.2019
%%
% Preamble
\tytul{Baby sienne i senne}{sł. A. Ziemianin, muz. K. Myszkowski}{Stare Dobre Małżeństwo}
\begin{text}
    Baby w biodrach mocarne\\
    odpoczywają przy sianie\\
    upałem całkiem zemdlone\\
    drzemią przed drugim śniadaniem

    \vin Siano pachnie\\
    \vin Siano kusi\\
    \vin Siano śpiewa\\
    \vin Siano mruczy\\
    \vin Siano kręci\\
    \vin Siano nęci\\
    \vin Siano budzi coś w pamięci

    Pierwszej chłop się ukazał\\
    w gorącym powietrzu\\
    Szepcze coś do ucha\\
    namawia do grzechu

    Druga z mrocznym aniołem\\
    w karty gra o życie\\
    a diabeł złote góry\\
    przynosi kobicie

    Trzeciej kosiarz młody\\
    całkiem zawrócił w głowie\\
    a ona panna jeszcze\\
    więc za kosiarzem w ogień

    \vin Siano pachnie...

    Baby choć w bodrach mocne\\
    i nie bite w ciemie\\
    z siana się wygrzebały\\
    i wróciły na ziemię

    A na ziemskim padole\\
    już baby wiedzą co grane\\
    więc żeby mieć święty spokój\\
    samy wypchały się sianem
\end{text}
\begin{chord}
    C\\
    d\\
    F\\
    G

    C\\
    F\\
    d\\
    G\\
    C\\
    F\\
    d G C
\end{chord}