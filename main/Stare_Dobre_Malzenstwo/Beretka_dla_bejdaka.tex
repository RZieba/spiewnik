%%
%% Author: bartek.rydz
%% 19.02.2019
%%
% Preamble
\tytul{Beretka dla Bejdaka}{sł. Adam Ziemianin, muz. Krzysztof Myszkowski}{Stare Dobre Małżeństwo}
\begin{text}
    Skrajem nieba szedł Bejdak\\
    Organki z odpustu same mu grały\\
    Kałużę żabom łyżką zamieszał\\
    Nie zdając sobie z tego sprawy

    Podobno ktoś widział jak do żab się łasił\\
    Na kolanach żeby mu kumkały\\
    Albo w pasiece do ula się spowiadał\\
    A pszczoły miód na serce mu lały

    Ktoś go poprosił by mu łąkę skosił\\
    A on zioła głaskał, tulił się do trawy\\
    Jakoś nie umiał z ludźmi żyć\\
    Raczej kumplował się z ptakami\\
    Kiedyś nad ranem z nimi odleciał\\
    Na niebieskie ptasie polany\\
    Jakoś nie umiał z ludźmi żyć

    Czasem wróblem wraca gdy Boga uprosi\\
    Z lotu ptaka chwilę u nas gości\\
    Boga słabość do niego jednaka\\
    Bo jak nie kochać takiego Bejdaka
\end{text}
\begin{chord}
    C\\
    F C\\
    d F\\
    G

    C\\
    F C\\
    d F\\
    G

    F\\
    C G\\
    d F\\
    G
\end{chord}