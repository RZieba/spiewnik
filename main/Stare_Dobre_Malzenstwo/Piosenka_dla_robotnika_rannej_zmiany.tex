%%
%% Author: bartek.rydz
%% 17.02.2019
%%
% Preamble
\tytul{Piosenka dla robotnika rannej zmiany}{sł. Edward Stachura, muz. Krzysztof Myszkowski}{Stare Dobre Małżeństwo}
\begin{text}
    Godzina słynna: piąta pięć\\
    Naciska budzik, dźwiga się\\
    Do kuchni drogę zna na pamięć\\
    Prowadzą go tam nogi same\\
    Pod kran pakuje śpiący łeb\\
    Przez chwilę jeszcze śpi jak w łóżku\\
    Dopóki nie posłyszy plusku\\
    I wtedy wreszcie budzi się

    Aniele Pracy - stróżu mój\\
    Jak ciężki robotnika znój

    Zbożowa kawa, smalec, chleb\\
    Salceson czasem, kiedy jest\\
    Do teczki drugie pcha śniadanie\\
    I teraz szybko na przystanek\\
    W tramwaju tłok i nie ma Boga\\
    Jest ramię w ramię, w nogę noga\\
    Kimanie na stojąco jest

    Aniele Pracy - stróżu mój\\
    Jak ciężki robotnika znój

    Przez osiem godzin praca wre\\
    Jak z bicza strzelił minął dzień\\
    Już w domu siedzi przed ekranem\\
    Na stole flaszka z marcepanem\\
    Dziś chłopcy grają ważny mecz\\
    Przez cały czas w ataku nasi

    Aniele Pracy - stróżu mój\\
    Jak ciężki robotnika znój

    Niech nas ukoi dobry sen\\
    Najlepsza w końcu jest to rzecz\\
    I co się śni? Podwyżka cen\\
    Aniele Pracy - stróżu mój\\
    Jak ciężki robotnika znój
\end{text}
\begin{chord}
    E\\
    E\\
    E\\
    E E^{7}\\
    A\\
    E\\
    H^{7}\\
    A E H^{7}
\end{chord}