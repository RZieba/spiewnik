\tytul{Pastuch}
{Jaromir Nohavica - Antoni Muracki}
{Jaromir Nohavica}
\begin{text}
Gdy byłem mały wciąż mi mówił tata\\
że jeszcze zrobi ze mnie adwokata\\
więc paragrafy musiałem wbijać do głowy\\
Taki adwokat grubą forsę kosi\\
siedzi w fotelu i dłubie palcem w nosie\\
a ja mu na to, że wolę wypasać krowy

Ja chciałbym\\
Mieć czapkę z pomponem z boku\\
jeść ulęgałki, pływać w potoku\\
i śpiewać przez cały dzień\\
refrenik ten, tak śpiewać\\
pam pam padam pam pada dam\\
pam pam padam pam pada dam\\
pam pa da da dam pa da da da-am

Stosy książek pod choinkę dawali mi\\
ale nadal nie umiałem odnaleźć w nich\\
prostej instrukcji – jak wypasa się krowy\\
Pytałem starszych, lecz każdy się śmiał\\
i telefon do lekarza podać mi chciał\\
i pytał czy poza tym w domu wszyscy są zdrowi

Ja chciałbym\\
Mieć czapkę z pomponem z boku\\
jeść ulęgałki, pływać w potoku\\
i śpiewać przez cały dzień\\
refrenik ten, tak śpiewać\\
pam pam padam pam pada dam\\
pam pam padam pam pada dam\\
pam pa da da dam pa da da da-am

Dziś choć podrosłem i swoje już wiem\\
parę rzeczy mogę zmienić, a paru nie\\
to gdy mi smutno w mokrej kładę się trawie\\
Z dłońmi za głową sobie leżę, a co!\\
gapię się w granatowe nieba tło,\\
gdzie wśród obłoków moje łaciate krowy się bawią

Ja chciałbym\\
Mieć czapkę z pomponem z boku\\
jeść ulęgałki, pływać w potoku\\
i śpiewać przez cały dzień\\
refrenik ten, tak śpiewać\\
pam pam padam pam pada dam\\
pam pam padam pam pada dam\\
pam pa da da dam pa da da da-am
\end{text}
\begin{chord}
\end{chord}
