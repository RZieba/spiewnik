%%
%% Author: bartek.rydz
%% 16.02.2019
%%
% Preamble
\tytul{Szkoda, że Cię tu nie ma}{sł. Josif Brodski, tłum. Stanisław Barańczak, muz. Kuba Blokesz}{Kuba Blokesz}
\begin{text}
    Szkoda, że cię tu nie ma,
    szkoda, kochanie.
    Siedziałabyś na sofie,
    ja - na dywanie.
    Chustka byłaby twoja,
    moja - kapiąca łza.
    Albo może na odwrót:
    płacz ty - pociecha - ja.

    Szkoda, że cię tu nie ma,
    szkoda, kochanie.
    Prowadząc wóz, dłoń kładłbym
    na twym kolanie,
    udając, że je mylę
    z dźwignią, gdy zmieniam bieg.
    Wabiłby nas nieznany
    lub właśnie znany brzeg.

    Szkoda, że cię tu nie ma,
    szkoda, kochanie.
    Srebrny księżyc na czarnym
    nieba ekranie
    na przekór astronomom
    oddawałbym co noc
    na żeton na automat,
    by usłyszeć twój głos.

    Szkoda, że cię tu nie ma,
    na tej półkuli -
    myślę, siedząc na ganku w letniej koszuli
    i z puszką "Heinekena".
    Zmierzch. Krzyk mew. Liści szmer.
    Co za zysk z zapomnienia,
    jeśli tuż po nim - śmierć?
\end{text}
\begin{chord}

\end{chord}